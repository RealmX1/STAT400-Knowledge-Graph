\documentclass{article}
\usepackage{amsmath}

\begin{document}

\section*{Poisson Distribution}

\textbf{Name:} Poisson Distribution

\textbf{Definition:} \\
The Poisson random variable models the number of events that occur in a fixed interval of time or space when events happen with a known constant mean rate and independently of the time since the last event.

\textbf{Poisson Distribution:} We say that \( X \) has the Poisson distribution with parameter:
\begin{itemize}
    \item \( \lambda > 0 \): the average number of events in the interval.
\end{itemize}

\textbf{Notation:} \\
\( X \sim \text{Poisson}(\lambda) \) if \( X \) takes values \( X := \{0, 1, 2, 3, \ldots\} = \mathbb{N} \) and the probability mass function of \( X \) is given as
\[
p_X(x) = \frac{\lambda^x e^{-\lambda}}{x!}, \quad x \in \{0, 1, 2, 3, \ldots\}
\]

\textbf{Aliases:} \\
Poisson Process

\end{document}
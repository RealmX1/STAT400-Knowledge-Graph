\documentclass{article}
\usepackage{amsmath}

\title{Geometric to Negative Binomial}
\author{}
\date{}

\begin{document}

\maketitle

\section*{Conversion from Geometric to Negative Binomial Distribution}
\textbf{Aliases}: None

\section*{Hypotheses/Assumptions}
\begin{itemize}
    \item The geometric distribution models the number of failures before the first success in a series of Bernoulli trials.
    \item The negative binomial distribution models the number of failures before achieving \( n \) successes in a series of Bernoulli trials.
    \item Assume independent and identically distributed trials with success probability \( p \).
\end{itemize}

\section*{Statement}
\begin{itemize}
    \item The geometric distribution is a special case of the negative binomial distribution where the number of required successes \( n = 1 \).
\end{itemize}

\section*{Explanation}
The probability mass function of the geometric distribution is
\[
P(X = k) = (1 - p)^k p,
\]
where \( k \) is the number of failures before the first success.

The probability mass function of the negative binomial distribution, representing the number of failures before \( n \) successes, is
\[
P(X = k) = \binom{k + n - 1}{k} (1 - p)^k p^n,
\]
where \( k \) is the number of failures and \( n \) is the number of successes required.

\end{document}

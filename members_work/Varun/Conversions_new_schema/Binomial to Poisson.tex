\documentclass{article}
\usepackage{amsmath}

\title{Binomial to Poisson}
\author{}
\date{}

\begin{document}

\maketitle

\section*{Conversion from Binomial to Poisson Distribution}
\textbf{Aliases}: None

\section*{Hypotheses/Assumptions}
\begin{itemize}
    \item The binomial distribution models the number of successes in a fixed number of independent Bernoulli trials.
    \item The Poisson distribution is a limiting case of the binomial distribution when the number of trials is large, and the success probability is small, while keeping the expected number of successes constant.
\end{itemize}

\section*{Statement}
\begin{itemize}
    \item The binomial distribution $B(n, p)$ converges to the Poisson distribution $P(\lambda)$ as $n \to \infty$ and $p \to 0$ such that $np = \lambda$, where $\lambda$ is the expected number of successes.
\end{itemize}

\section*{Explanation}
When the number of trials $n$ is very large and the probability of success $p$ is very small, the binomial probability mass function
\[
P(X = k) = \binom{n}{k} p^k (1 - p)^{n - k}
\]
can be approximated by the Poisson probability mass function
\[
P(X = k) = \frac{\lambda^k e^{-\lambda}}{k!}
\]
where $\lambda = np$.

\end{document}

\documentclass{article}
\usepackage{amsmath}

\title{Poisson to Self}
\author{}
\date{}

\begin{document}

\maketitle

\section*{Poisson to Self Transformation}
In the diagram, there is a self-relation for the \textbf{Poisson} distribution indicated by the transformation:
\[
\sum X_i
\]
This implies that the \textbf{sum of independent and identically distributed (i.i.d.) Poisson random variables} also follows a Poisson distribution.

\section*{Explanation}
If \( X_1, X_2, \dots, X_n \) are independent Poisson random variables with parameters \( \lambda_1, \lambda_2, \dots, \lambda_n \), then the sum \( X = X_1 + X_2 + \dots + X_n \) is also a Poisson random variable with parameter:
\[
\lambda = \lambda_1 + \lambda_2 + \dots + \lambda_n.
\]

In the special case where all the \( X_i \)'s have the same parameter \( \lambda \), the sum:
\[
X = X_1 + X_2 + \dots + X_n
\]
follows a Poisson distribution with parameter \( n\lambda \).

This property is a key characteristic of the Poisson distribution, and it is represented in the diagram by the notation \( \sum X_i \), indicating that the sum of i.i.d. Poisson random variables is still Poisson-distributed.

\end{document}

\documentclass{article}
\usepackage{amsmath}

\title{Negative Binomial to Poisson}
\author{}
\date{}

\begin{document}

\maketitle

\section*{Conversion from Negative Binomial to Poisson Distribution}
\textbf{Aliases}: None

\section*{Hypotheses/Assumptions}
\begin{itemize}
    \item The negative binomial distribution models the number of failures before achieving \( n \) successes in a sequence of Bernoulli trials.
    \item The Poisson distribution can be seen as the limiting case of the negative binomial distribution when the number of trials increases to infinity while keeping the mean constant.
\end{itemize}

\section*{Statement}
\begin{itemize}
    \item The negative binomial distribution \( NB(r, p) \) converges to the Poisson distribution \( P(\lambda) \) as the number of trials tends to infinity, with a small success probability, provided that \( r \times p = \lambda \), where \( \lambda \) represents the expected number of events.
\end{itemize}

\section*{Explanation}
The probability mass function of the negative binomial distribution for achieving \( n \) successes is
\[
P(X = k) = \binom{k + n - 1}{k} (1 - p)^k p^n,
\]
where \( k \) represents the number of failures, \( n \) is the required number of successes, and \( p \) is the probability of success in each trial.

As \( n \to \infty \) and \( p \to 0 \), with \( \lambda = n \cdot p \), this distribution converges to the Poisson distribution:
\[
P(X = k) = \frac{\lambda^k e^{-\lambda}}{k!}.
\]

\end{document}

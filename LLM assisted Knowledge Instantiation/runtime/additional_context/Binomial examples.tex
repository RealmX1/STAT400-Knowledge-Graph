\documentclass{article}
\usepackage{amsmath}

\begin{document}

\section*{Binomial Distribution}

\section*{Examples}

\begin{enumerate}
    \item Suppose we flip a fair coin \( n \) times. Let \( X \) represent the number of times we get heads. This experiment can be modeled by \( X \sim \text{Binom}(n, p) \), where
    \[
    p = 0.5
    \]
    and \( p_X(k) = P(X = k) \) represents the probability of obtaining exactly \( k \) heads in \( n \) flips.

    \item Suppose a factory produces light bulbs, and each bulb has a probability \( p \) of being defective. Let \( X \) represent the number of defective bulbs in a sample of \( n \) bulbs. This scenario can be modeled by \( X \sim \text{Binom}(n, p) \), where
    \[
    p = \text{probability that a bulb is defective}.
    \]
    In this setting, \( p_X(k) = P(X = k) \) is the probability of finding exactly \( k \) defective bulbs in a sample of \( n \) bulbs.

    \item Suppose a survey asks \( n \) people whether they prefer tea over coffee, with each person having an independent probability \( p \) of preferring tea. Let \( X \) represent the number of people who prefer tea. This can be modeled by \( X \sim \text{Binom}(n, p) \), where
    \[
    p = \text{probability of a person preferring tea}.
    \]
    In this setting, \( p_X(k) = P(X = k) \) is the probability of exactly \( k \) people preferring tea out of \( n \) surveyed.
\end{enumerate}

\end{document}
\documentclass[11pt,oneside]{book}

%%%%%%%%%%%%%%%%%%%%%%%%%%%%%%%
%[itemsep=0pt, topsep=1pt, partopsep=0pt,label=(\alph*)]
%%%%%%%%%%%%%%%%%%%%%%%%%%%%%%%

%%%%%%%%%%%%%%Include Packages%%%%%%%%%%%%%%%%%%%%%%%%%%

\usepackage{xcolor}
\usepackage{tikz,xcolor,color}
\usepackage{mathtools}
\usepackage[legalpaper, margin=0.8in]{geometry}
\usetikzlibrary{fit}
\usepackage{amsmath}
\usepackage{amssymb}
\usepackage{paralist}
\usepackage{rsfso}
\usepackage{amsthm}
\usepackage{wasysym}
\usepackage[inline]{enumitem}   
\usepackage{hyperref}
\usepackage{tocloft}
\usepackage{wrapfig}
\usepackage{titlesec}
\usepackage[makeroom]{cancel}
\usepackage [english]{babel}
\usepackage [autostyle, english = american]{csquotes}
\usetikzlibrary{trees}
\usepackage[latin1]{inputenc}
\tikzstyle{level 1}=[level distance=3.5cm, sibling distance=4cm]
\tikzstyle{level 2}=[level distance=3.5cm, sibling distance=3cm]
\usepackage{eucal}
\usepackage[T1]{fontenc}
\usepackage{mathtools}
\usepackage[thinc]{esdiff}
\usepackage[toc,page]{appendix} %---------------%%%%%%%%%%
% Define styles for bags and leafs
\tikzstyle{bag} = [text width=3em, text centered]
\tikzstyle{end} = [circle, minimum width=3pt,fill, inner sep=0pt]
\MakeOuterQuote{"}
%%%%%%%%%%%%%%%%%%%%%%%%%%%%%%%%%%%%%%%%%%%%%%%%%%%%%%%%


%%%%%%%%%%%%%%%%%%%%%%%%%%%%%%%%%%%%%%%%%%%%%%%%%%%%%%%%%%%%%%%%
%%%%%%%%%%%%%%%%%%% FOR EXAMPLES %%%%%%%%%%%%%%%%%%%%
%%%%%%%%%%%%%%Include Packages%%%%%%%%%%%%%%%%%%%%%%%%%%
\usepackage{xcolor}
\usepackage{colortbl}
\usepackage{tikz}
\usepackage{mathtools}
\usepackage[legalpaper, margin=0.8in]{geometry}
\usepackage{amsmath}
\usepackage{amsthm}
\usepackage{amssymb}
\usepackage{rsfso}
\usepackage{wasysym}
\usepackage{hyperref}
\usetikzlibrary{matrix, calc, arrows,
                arrows.meta, fit,
                positioning, quotes,
                shapes.geometric}


%%%%%%%%%%%%%%%Color%%%%%%%%%%%%%%%%%%%%%%%%%%
\definecolor{gray75}{gray}{0.75}
\definecolor{yellow}{RGB}{255,255,177}
\definecolor{pink}{RGB}{250,204,224}
%%%%%%%%%%%%%%%%%%%%%%%%%%%%%%%%%%%%%%%%%%%%%%%%%%%%%%%%

%%%%%%%%%%%%%%%%%Theorem environments%%%%%%%%%%%%%%%%%%%
\newtheoremstyle{break}
  {\topsep}{\topsep}%
  {\itshape}{}%
  {\bfseries}{}%
  {\newline}{}%
\newtheoremstyle{newStyle}
  {\topsep}{\topsep}%
  {\rmfamily}{}%
  {\bfseries}{}%
  {\newline}{}%             % Theorem head spec
  \theoremstyle{newStyle}
\newtheorem{thm}{Theorem}[chapter]
\newtheorem{lem}{Lemma}[thm]
\newtheorem{axiom}[thm]{Axiom}
\newtheorem{prop}[lem]{Proposition}
\newtheorem{cor}[lem]{Corollary}
\newtheorem{defn}[thm]{Definition}
%%%%%%%%%%%%%%%%%%%%%%%%%%%%%%%%%%%%%%%%%%%%%%%%%%%%%%



\newcommand{\R}{\mathbb{R}}
\newcommand{\N}{\mathbb{N}}
\newcommand{\Z}{\mathbb{Z}}
\newcommand{\Q}{\mathbb{Q}}
\newcommand{\spa}{\text{span}}
\newcommand{\pd}{\partial}
\newcommand{\that}[1]{\widetilde{#1}}
\newcommand{\vmat}[1]{\begin{vmatrix} #1 \end{vmatrix}}
\newcommand{\bmat}[1]{\begin{bmatrix} #1 \end{bmatrix}}
\newcommand{\pmat}[1]{\begin{pmatrix} #1 \end{pmatrix}}
\newcommand*{\Perm}[2]{{}^{#1}\!P_{#2}}
\newcommand*{\Comb}[2]{{}^{#1}C_{#2}}
\newcommand*{\dr}[5]{\draw ({#1}.east) --+ ({#4},{#5}) node[right] ({#2}) {{#3}};}
%%%%%%%%%%%%%%%%%%%%%%%%%%%%%%%%%%%%%%%%%%%%%%%%%%%%%%%%%%%%%%%%
%%%%%%%%%%%%%%%%%%%%%%%%%%%%%%%%%%%%%%%%%%%%%%%%%%%%%%%%%%%%%%%%



\newcommand{\note}{\color{red}Note: \color{black}}
\newcommand{\remark}{\color{blue}Remark: \color{black}}
\newcommand{\example}{\color{purple}Example: \color{black}}
\newcommand{\exercise}{\color{cyan}Exercise: \color{black}}





%%%%%%%%%%%%%%%Chapter Setting%%%%%%%%%%%%%%%%%%%%%%%%%%
\definecolor{gray75}{gray}{0.75}
\newcommand{\hsp}{\hspace{20pt}}
\titleformat{\chapter}[hang]{\Huge\bfseries}{\thechapter\hsp\textcolor{gray75}{$\mid$}\hsp}{0pt}{\Huge\bfseries}
\definecolor{yellow}{RGB}{255,255,177}
\definecolor{pink}{RGB}{250,204,224}
%%%%%%%%%%%%%%%%%%%%%%%%%%%%%%%%%%%%%%%%%%%%%%%%%%%%%%%%

%%%%%%%%%%%%%%%%%Theorem environments%%%%%%%%%%%%%%%%%%%
\newtheoremstyle{break}
  {\topsep}{\topsep}%
  {\itshape}{}%
  {\bfseries}{}%
  {\newline}{}%
\newtheoremstyle{newStyle}
  {\topsep}{\topsep}%
  {\rmfamily}{}%
  {\bfseries}{}%
  {\newline}{}%             % Theorem head spec
  \theoremstyle{newStyle}

\newtheorem{corL}{Corollary}[lem]
\newtheorem{corT}[lem]{Corollary}


\newtheorem{ex}{Example}[section]
\newenvironment{indEnv}[1][Proof]
  {\proof[#1]\leftskip=1cm\rightskip=1cm}
  {\endproof}
%%%%%%%%%%%%%%%%%%%%%%%%%%%%%%%%%%%%%%%%%%%%%%%%%%%%%%

%%%%%%%%%%%%%%%%%%%%%%%Integral%%%%%%%%%%%%%%%%%%%%%%%
\def\upint{\mathchoice%
    {\mkern13mu\overline{\vphantom{\intop}\mkern7mu}\mkern-20mu}%
    {\mkern7mu\overline{\vphantom{\intop}\mkern7mu}\mkern-14mu}%
    {\mkern7mu\overline{\vphantom{\intop}\mkern7mu}\mkern-14mu}%
    {\mkern7mu\overline{\vphantom{\intop}\mkern7mu}\mkern-14mu}%
  \int}
\def\lowint{\mkern3mu\underline{\vphantom{\intop}\mkern7mu}\mkern-10mu\int}
%%%%%%%%%%%%%%%%%%%%%%%%%%%%%%%%%%%%%%%%%%%%%%%%%%%%%%




\newcommand{\A}{\mathcal{A}}
\newcommand{\D}{\mathcal{D}}
\newcommand{\J}{\mathcal{J}}
\newcommand{\T}{\mathcal{T}}
\newcommand{\Td}{\mathcal{T}_d}
\newcommand{\X}{\mathcal{X}}
\newcommand{\Complex}{\mathbb{C}}
\newcommand{\Power}{\mathcal{P}}
\newcommand{\ee}{\cdot 10}

\newcommand{\rref}{\xrightarrow{\text{row\ reduce}}}




%%%%%%%%%%%%table of contents%%%%%%%%%%%%%%%%%%%%%%%%%%%%
\setlength{\cftchapindent}{0em}
\setlength{\cftsecindent}{2em}


\renewcommand\cfttoctitlefont{\hfill\Large\bfseries}
\renewcommand\cftaftertoctitle{\hfill\mbox{}}

\setcounter{tocdepth}{2}
%%%%%%%%%%%%%%%%%%%%%%%%%%%%%%%%%%%%%%%%%%%%%%%%%%%%%%%%%


%%%%%%%%%%%%%%%%%%%%%Footnotes%%%%%%%%%%%%%%%%%%%%%%%%%%%
\newcommand\blfootnote[1]{%
  \begingroup
  \renewcommand\thefootnote{}\footnote{#1}%
  \addtocounter{footnote}{-1}%
  \endgroup
}
%%%%%%%%%%%%%%%%%%%%%%%%%%%%%%%%%%%%%%%%%%%%%%%%%%%%%%%%%

\makeatletter
\def\@seccntformat#1{%
  \expandafter\ifx\csname c@#1\endcsname\c@section\else
  \csname the#1\endcsname\quad
  \fi}
\makeatother


%%%%%%%%%%%%%%%%%%%%%%%%%%%%%%%%%%%Enumerate%%%%%%%%%%%%%%
\makeatletter
% This command ignores the optional argument 
% for itemize and enumerate lists
\newcommand{\inlineitem}[1][]{%
\ifnum\enit@type=\tw@
    {\descriptionlabel{#1}}
  \hspace{\labelsep}%
\else
  \ifnum\enit@type=\z@
       \refstepcounter{\@listctr}\fi
    \quad\@itemlabel\hspace{\labelsep}%
\fi}
\makeatother
\parindent=0pt
%%%%%%%%%%%%%%%%%%%%%%%%%%%%%%%%%%%%%%%%%%%%%%%%%%%%%%%%%%
\tikzset{filled/.style={fill=circle area, draw=circle edge, thick},
    outline/.style={draw=circle edge, thick}}

%%%%%%%%%% Typesetting end-of-chapter exercises %%%%%%%%%%%
% Typesetting exercises
\newcounter{exercisenumber}
% Begin exercises
\newcommand{\thischapterexercises}{
    \section{Exercises for Chapter \thechapter}
    \setcounter{exercisenumber}{1}
}
% Exercise item
\newcommand{\exerciseitem}{
    \textbf{Exercise \thechapter.\arabic{exercisenumber}.}
    \stepcounter{exercisenumber}
}
\title{\Huge \textbf{Introduction to Probability and Statistics} 
\vspace{0.5in}

\Large Textbook for STAT/DATA 400 at University of Maryland, College Park
\footnote{Supported by UMD TLTC 2024 Teaching Innovation Grant\\
\url{https://tltc.umd.edu/2024-teaching-innovation-grant-awardees}}}


\date{}
\author{\textsc{Jonathan Fernandes} and  \textsc{Mestiyage Gunatilleka}}
%\subtitle{A Text for STAT/DATA400 at University of Maryland}

\begin{document}
\newpage

\maketitle
\tableofcontents
\addtocontents{toc}{~\hfill\textbf{Page}\par}

\vspace*{\fill}

\newpage
\chapter[Probability]{Probability}
\section{Sets, Experiment, and Probability}
\begin{defn}[Experiment]
An \textbf{Experiment} is a repeatable task with well defined outcomes.
\end{defn}

\begin{defn} [Sample Space]
The \textbf{Sample Space} for an experiment is the set of all possible outcomes of that experiment, usually denoted by $\mathcal{S}$
\end{defn}

\begin{defn}[Event]
Any subset $E$ of the sample space $\mathcal{S}$ attached to an experiment will be called an \textbf{Event} associated to the experiment. 
\end{defn}
To say that \textbf{an event $E$ has happened} means "we performed the experiment, and the outcome of this experiment was in the set $E$". 

\begin{defn}[Simple Event]
An event is called a  simple event if it has exactly one outcome in it.  
\end{defn}

\begin{ex}
    Experiments involving coin tosses:
    \begin{enumerate}
\item[(a).] Experiment: Toss a coin once. \\
All possible outcomes are contained in the set $\{H,T\}$, with $H$ representing obtaining a head and $T$ representing obtaining a tail. The total number of outcomes is 2. 
\item[(b).] Experiment: Toss the coin two times.\\
All possible outcomes are contained in the set $\{HH, HT, TH, TT\}$. Number of outcomes is 4. 
\item[(c).] Experiment: Toss the coin 10 times.\\
All possible outcomes are contained in the set $\{HHH\cdots H,\ HTHH\cdots H,\ \cdots\}$. Number of outcomes is $2^{10}=1024$. 
\end{enumerate}
\end{ex}

\begin{ex}
        Experiments involving die rolls:
\begin{enumerate}
\item[(a).] Roll a six-sided die once. 
Outcomes are in the set $\{1,2,3,4,5,6\}$.
\item[(b).] Roll the die two times. Outcomes are in the set
\begin{align*}
\left\{
\begin{matrix}
(1,1), & (1,2), & (1,3), & \cdots , & (1,6),\\
(2,1), & (2,2), & (2,3), & \cdots , & (2,6),\\
\vdots & \vdots & \vdots & \ddots  & \vdots\\
(6,1), & (6,2), & (6,3), & \cdots , & (6,6)
\end{matrix}
\right\}\ .
\end{align*}
\end{enumerate}
\end{ex}





\begin{ex}
    Consider the following two step experiment: In step 1, we toss a coin; In step 2, we roll a 6-sided die. All possible outcomes are contained in the set
\begin{align*}
\left\{
\begin{matrix}
(H,1), &(H,2), &(H,3),& (H,4),& (H,5),& (H,6), \\
(T,1),& (T,2), &(T,3), &(T,4),& (T,5), &(T,6) \\
\end{matrix}
\right\}\ .
\end{align*}
\end{ex} 

\begin{ex}
    We can perform an experiment where we pick a random student from campus, and ask if they have walked more than 3000 steps today. All possible outcomes of this experiment are contained in the set $\{``\text{Yes}",\,``\text{No}"\}$.
\end{ex}

\begin{ex}
    We can perform an experiment where we pick a student from campus and measure their height. All possible outcomes are contained in the set of positive real numbers, that is $(0, \infty)$.
\end{ex}
 


Our goal is to assign probabilities or chance to events $E$ corresponding to an experiment. Since events and sample spaces are described as sets and subsets, we will first need to remind ourselves of how to work with these mathematical objects.  

\section[Set Theory]{Basic Set Theory}
We will need to understand some basics of sets and set operations. We define the operations along with the visual of the the operation using Venn Diagrams.

\begin{defn}[Empty Set]
    An an empty set is a set with nothing in it, usually denoted as $\emptyset$.
\end{defn}

Let $A,B$ be two sets.\\
\begin{defn}[Containment/Subset]
    We say $A$ is a \textbf{subset} of $B$ if \textit{$x\in A$ \textbf{implies} $x\in B$}, denoted as $A\subset B$.\\
\end{defn}

\begin{defn}[Equality]
    We say $A$ \textbf{equals} to $B$ if \textit{$A\subset B$ \textbf{and} $B\subset A$}.  
\end{defn}

\begin{defn}[Complement]
    The \textbf{complement} of A is the set $A^c:=\{x:x\notin A\}$
    \begin{center}
\begin{tikzpicture}
    % Definition of circles
    \def\firstcircle{(0,0) circle (1.5cm)}
    \def\secondcircle{(0:2cm) circle (1.5cm)}
    %
    \colorlet{circle edge}{black!50}
    \colorlet{circle area}{gray!20}
    \begin{scope}[local bounding box = orScope]
      \begin{scope}
        \clip \secondcircle;
        \fill[filled] \secondcircle;
    \end{scope}
    \draw[outline] \firstcircle node {$A$};
    \draw[outline] \secondcircle node {$B$};
    \node[anchor=south] at (orScope.north) {$U$};  
    \end{scope}
    \node[fit=(orScope), fill=gray!20] {};
    \node[fit=(orScope), draw] {};
     \begin{scope}
        \clip \secondcircle;
        \fill[filled] \secondcircle;
    \end{scope}
    \begin{scope}
        \clip \firstcircle;
        \fill[white] \firstcircle;
    \end{scope}
    \draw[outline] \firstcircle node {$A$};
    \draw[outline] \secondcircle node {$B$};
    \node[anchor=south] at (orScope.north) {$U$};  
\end{tikzpicture}
\end{center}
\end{defn}

\begin{defn}[Union]
    The \textbf{union} of A and B is the set $A\cup B:=\{x:x\in A \text{ or }x\in B\}$
    \begin{center}
\begin{tikzpicture}
    % Definition of circles
    \def\firstcircle{(0,0) circle (1.5cm)}
    \def\secondcircle{(0:2cm) circle (1.5cm)}
    %
    \colorlet{circle edge}{black!50}
    \colorlet{circle area}{gray!20}
    %
    \begin{scope}[local bounding box = orScope]
    \draw[filled] \firstcircle node {$A$}
                  \secondcircle node {$B$};
    \node[anchor=south] at (orScope.north){$U$};
    \end{scope}
    \node[fit=(orScope), draw] {};% The frame around the scope
\end{tikzpicture}
\end{center}
\end{defn}

\begin{defn}[Intersection]
    The \textbf{intersection} of A and B is the set $A\cap B:=\{x:x\in A\text{ and }x\in B\}$
    \begin{center}
\begin{tikzpicture}
    % Definition of circles
    \def\firstcircle{(0,0) circle (1.5cm)}
    \def\secondcircle{(0:2cm) circle (1.5cm)}
    %
    \colorlet{circle edge}{black!50}
    \colorlet{circle area}{gray!20}
    \begin{scope}[local bounding box = orScope]
    \begin{scope}
        \clip \firstcircle;
        \fill[filled] \secondcircle;
    \end{scope}
    \draw[outline] \firstcircle node {$A$};
    \draw[outline] \secondcircle node {$B$};
    \node[anchor=south] at (orScope.north) {$U$};
    \end{scope}
    \node[fit=(orScope), draw] {};
\end{tikzpicture}
\end{center}
\end{defn}

\begin{defn}[Difference]
   The \textbf{difference} of set $A$ from set $B$ is defined as $A\setminus B:=\{ x: x\in A\; \text{and}\; x\notin B\}$.
   This is the same as $A\setminus B := A\cap B^c$. 
    \begin{center}
\begin{tikzpicture}
    % Definition of circles
    \def\firstcircle{(0,0) circle (1.5cm)}
    \def\secondcircle{(0:2cm) circle (1.5cm)}
    %
    \colorlet{circle edge}{black!50}
    \colorlet{circle area}{gray!20}
    \begin{scope}[local bounding box = orScope]
    \begin{scope}
        \clip \firstcircle;
        \fill[filled] \firstcircle;
    \end{scope}
    \begin{scope}
        \clip \secondcircle;
        \fill[white] \secondcircle;
    \end{scope}
    \draw[outline] \firstcircle node {$A$};
    \draw[outline] \secondcircle node {$B$};
    \node[anchor=south] at (orScope.north) {$U$};
    \end{scope}
    \node[fit=(orScope), draw] {};
\end{tikzpicture}
\end{center}
\end{defn}









\begin{defn}[Disjoint Sets/Mutually Exclusive]
$A$ and $B$ are said to be \textbf{disjoint} if $A\cap B=\emptyset$.
\end{defn}

We will often have to work with an infinite collection of sets, to do so, we will first need to label the sets using an "indexing set". Given a set $\Gamma$, a collection of sets indexed by $\Gamma$ will be denoted by $\{A_\alpha\}_{\alpha\in \Gamma}$. 
\\
A set is \textbf{countable} if there is a one-to-one correspondence with its elements, and the set of natural number $\mathbb{N}= \{1, 2, 3, 4, \dots\}$. 
\\

Let $\{A_i\}_{i\in\mathbb{N}}$ be a countable collection of sets. 

\begin{defn}[Pairwise Disjoint Collection]
We say $\{A_i\}_{i\in\mathbb{N}}$ is a collection of pairwise disjoint sets if $A_i\cap A_j=\emptyset$ for all $i\neq j$
\end{defn}

\begin{defn}[Partition]
A collection $\{A_i\}_{i\in\mathbb{N}}$ is said to be a \textbf{partition} of $\mathcal{S}$ if $\{A_i\}_{i\in\mathbb{N}}$ are pairwise disjoint \textbf{and} $\mathcal{S}=\bigcup_{i=1}^{\infty}A_i$
\end{defn}

\begin{defn}[Countable Union/Intersection]
Let $\{A_i\}_{i\in\mathbb{N}}$ be a countable collection of subsets of $\mathcal{S}$.
\\
The \textbf{union} of $\{A_i: i \in \mathbb{N}\}$ is defined as \begin{align*}
\bigcup_{i\in \mathbb{N}}A_i &:=\{x : x\in A_i \;\text{ for some }\; i  \in \mathbb{N}\}
\end{align*}
The \textbf{intersection} of $\{A_i: i \in \mathbb{N}\}$  is defined as \begin{align*}
\bigcap_{i\in \mathbb{N}}A_i&=\{x : x\in A_i\; \text{ for all }\; i \in \mathbb{N}\}
\end{align*}
\end{defn}


\subsection[Axioms of Probability]{Axioms of Probability}
Let $\mathcal{S}$ be the sample space of an experiment and let $E\subset \mathcal{S}$ be and event. We are interested in assigning a number to $E$ that quantifies the chance/probability that $E$ occurs. We will denote the probability of $E$ as $P(E)$.
\\
The mathematical procedure to assign probabilities to events is to define a function on the "event space". Assigning probabilities to every possible event can be tedious, and in some case it might not be possible. As a result, we will first need to identify a collection of events that we consider "interesting" and to which we want to assign probabilities. Such a collection of events is called a "sigma algebra". 

\begin{defn}[Sigma Algebra]
Let $\mathcal{S}$ be a sample space of an experiment. A collection, $\mathcal{B}$, of subset of $\mathcal{S}$ is called a \textbf{sigma algebra} is it satisfies the following conditions:
\begin{enumerate}
    \item $\emptyset \in \mathcal{B}$
    \item If $A \in \mathcal{B}$, then $A^c \in \mathcal{B}$
    \item If $\{A_i: i \in \mathbb{N}\}$ is countable collection such that $A_i \in \mathcal{B}$ for all $i$, then $\bigcup_{i\in \mathbb{N}}A_i \in \mathcal{B}$
\end{enumerate}
\end{defn}
Note that there are multiple sigma algebras that can be associated to a sample space. The trivial sigma algebra is $\{\emptyset, \mathcal{S}\}$ - the smallest possible sigma algebra. The power set $\mathcal{P}(\mathcal{S})$ of $\mathcal{S}$ also satisfies the axioms to be a sigma algebra, and this is the largest possible sigma algebra associated to $\mathcal{S}$. All other sigma algebra are somewhere in between these two extreme examples. We can use sub-setting as a natural way to define a partial order on all possible sigma algebras. 
\\

A probability function assigns probabilities to events in the sigma algebra as follows.

\begin{defn}[Probability Function]
    Consider the pair $(\mathcal{S}, \mathcal{B})$, where $\mathcal{S}$ is the sample space of an experiment and $\mathcal{B}$ is a sigma algebra associated to $\mathcal{S}$. 
    \\
    A probability function for the pair $(\mathcal{S}, \mathcal{B})$, is a function 
    $$P: \mathcal{B} \longrightarrow \mathbb{R}$$
    satisfying the following axioms:
    \begin{enumerate}
    \item (finite measure) $P(\mathcal{S})=1$
    \item (positivity) $P(A)\geq 0$ for all $A \in \mathcal{B}$
    \item (countable additivity) For $A_1,A_2,A_3,\cdots$, the collection of pairwise disjoint subsets of $\mathcal{S}$ in $\mathcal{B}$, we must have
    $$P\left(\bigcup_{i\in \mathbb{N}}A_i\right)=\sum_{i=1}^{\infty}P(A_i)$$ 
    \end{enumerate}  
\end{defn}

\begin{thm}[Properties of a Probability Function]
Suppose $P$ is a probability function.
\begin{enumerate}[itemsep=0pt, topsep=1pt, partopsep=0pt,label=(\roman*)]
\item $P(\emptyset)=0$
\item If $A \subset B$, then $P(A)\le P(B)$
\item $P(A^c)=1-P(A)$
\item $P(A\setminus B)=P(A)-P(A\cap B)$
\item $P(A\cup B) =P(A)+P(B)-P(A\cap)B$
\end{enumerate}
\end{thm}
Defining a probability function can often times be a tedious task, as the number of conditions needed to be checked grow exponentially (as a function of the sigma algebra). 
\\
When the sample space $\mathcal{S}$ is countable, it is possible to define a probability function on the power set of $\mathcal{S}$ using the following theorem.

\begin{thm}
    Suppose the sample space of an experiment is countable, listed as, $\mathcal{S} = \{s_i\}_{i\in \mathbb{N}} = \{s_1, s_2, s_3, \dots, \}$. Let $\mathcal{B}:=\mathcal{P}(\mathcal{S})$ be the powerset of $\mathcal{S}$. 
    \\
    Suppose $\{p_i\}_{i\in \mathbb{N}}$ is a sequence of real numbers satisfying:
    $$ p_i \ge 0 \quad \forall i, \quad \quad \sum_{i=1}^\infty p_i = 1$$
    define $P:\mathcal{B} \rightarrow [0,1]$
    as follows:
    \begin{enumerate}
        \item For any $s_i \in \mathcal{S}$, $$P(\{s_i\}) = p_i$$
        \item For any $E \in \mathcal{B}$
        $$P(E) = \sum_{s\in E} P(\{s\})$$
    \end{enumerate}
    The function $P$ defines a probability function on $\mathcal{B}$. 
\end{thm}

Using the theorem above, we can assign probability functions experiments with countable sample space, by assigning probabilities just to the simple events. 

\begin{ex}
    Consider the experiment consisting of tossing a coin two times. The sample space is $S = \{HH, HT, TH, TT\}$. Take the largest possible sigma algebra, that is the power set of $S$, denoted as $\mathcal{P}(S)$, which contains $|\mathcal{P}(S)| = 2^4 = 16$ elements. In this experiment, if the coin is a fair coin, we can assign the following probability function $P: \mathcal{P}(S)\to \R$:
\begin{center}
\begin{tabular}{|c|c|}
\hline \rowcolor{lightgray}
\textbf{event $E$} & $P(E)$\\
\hline
$\emptyset$ & 0 \\
\hline
$\{HH\}$ & 1/4\\
\hline
$\{HT\}$ & 1/4\\
\hline
$\{TT\}$ & 1/4\\
\hline
$\{TH\}$ & 1/4\\
\hline
$\{HH, HT\}$ & 1/4+1/4 = 1/2\\
\hline
\vdots & \vdots \\
\hline
$\{HH, HT, TH\}$ & 1/4+1/4+1/4 = 3/4\\
\hline 
\vdots & \vdots \\
\hline
$\{HH, HT, TH, TT\}$ & 1\\
\hline
\end{tabular}\,.
\end{center}
In the case where the coin is not a fair coin, the probability function can be defined in the following way:
\begin{center}
\begin{tabular}{|c|c|}
\hline \rowcolor{lightgray}
\textbf{event $E$} & $P(E)$\\
\hline
$\emptyset$ & 0 \\
\hline
$\{HH\}$ & 1/3\\
\hline
$\{HT\}$ & 1/3\\
\hline
$\{TT\}$ & 1/3\\
\hline
$\{TH\}$ & 0\\
\hline
$\{HH, TH\}$ & 1/3+0 = 1/3\\
\hline
\vdots & \vdots \\
\hline
$\{HH, HT, TT\}$ & 1/3+1/3+1/3 = 1\\
\hline 
\vdots & \vdots \\
\hline
$\{HH, HT, TH, TT\}$ & 1\\
\hline
\end{tabular}\,.
\end{center}
Similarly, the following probability function is also allowed:
\begin{center}
\begin{tabular}{|c|c|}
\hline \rowcolor{lightgray}
\textbf{event $E$} & $P(E)$\\
\hline
$\emptyset$ & 0 \\
\hline
$\{HH\}$ & 1/8\\
\hline
$\{HT\}$ & 1/8\\
\hline
$\{TT\}$ & 3/8\\
\hline
$\{TH\}$ & 3/8\\
\hline
$\{HH, TH\}$ & 1/8+3/8 = 1/2\\
\hline
\vdots & \vdots \\
\hline
$\{HH, HT, TT\}$ & 1/8+1/8+3/8 = 5/8\\
\hline 
\vdots & \vdots \\
\hline
$\{HH, HT, TH, TT\}$ & 1\\
\hline
\end{tabular}\,.
\end{center}
Notice that the probabilities of events like $\{HH, TH\}$ and $\{HH, HT, TT\}$ are completely determined by the probabilities of the four simple events $\{HH\}$, $\{HT\}$, $\{TH\}$ and $\{TT\}$. While for event $E$, we note that $P(E)$ must be non-negative. We also observe that $P(S) = 1$, and as $S = S\cup \emptyset$, then $P(S) = P(S\cup \emptyset) = P(S) + P(\emptyset) = 1$, from which we deduce that we must have $P(\emptyset) = 0$. The tables shown above are called the distribution tables, and we see that there can be different distribution tables for the same sample space. 
\end{ex}

\section{Conditional Probability and Independence}
If the probability function attached to an experiments must be useful (say to draw some conclusions about events associated to the experiment), we will need to incorporate the features of the experiment as in our pursuit to calculate probability functions. 

\begin{defn}[Conditional Event]
    Suppose $E$ and $F$ are two events, the conditional event $E\mid F$ - verbalized as "E given F" - is the event that "E happens given that F has already happened"
\end{defn}

Given two events $E$ and $F$ we can consider the following events:
\begin{enumerate}
    \item $E\cap F$: both events $E$ and $F$ happen. 
    \item $E\mid F$: $E$ happens given that $F$ has happened.
    \item $F\mid E$: $F$ happens given that $E$ has happened.
\end{enumerate}
There is an interesting relationship between these events given by 
\begin{defn}[Multiplication Principle]
    Suppose $E$ and $F$ are two events then we define:
    $$P(E\cap F) = \begin{cases} P(E)P(F\mid E) \\
    P(F)P(E\mid F)
    \end{cases}$$
\end{defn}
The term $P(F\mid E)$ is called the conditional probability of the event "F given E", in fact we use the multiplication principle to formally define conditional proability
\begin{defn}[Conditional Probability]
    Suppose $E$ and $F$ are two events then we define:
    $$P(E\mid F) = \frac{P(E\cap F)}{P(F)} $$
\end{defn}

The multiplication principle informs us that the probability of the simultaneous occurrence of event $E$ and event $F$, that is $P(E\cap F)$, depends not only on information about the individual occurrence (with no additional information about the other event) of $E$ (that is $P(E)$) and $F$ (that is $P(F)$), \textbf{but} also the interaction/influence of how the event $E$ affects the occurrence of event $F$ (given by the conditional probability $P(F \mid E)$) or how the event $F$ affects the occurrence of event $E$ (given by the conditional probability $P(E \mid F)$).
\\

This naturally leads to the notion of independence, which gives the probabilistic description of what it means for the lack of interaction between two events. 
\\

\begin{defn}[Independent Events]
    We say two events $E$ and $F$ are independent if $$P(E\cap F) = P(E) P(F)$$
    Equivalently, $E$ and $F$ are independent if
    $$ P(E\mid F) = E(E) \quad \text{or}\quad  P(F\mid E) = P(F)$$
\end{defn}

Observe that, $P(E \mid F) = E(E)$, tells us that the fact that $F$ happened did not affect the probability of $E$ happening. 
\\


An experiment that can be broken down into a sequence of steps can be represented as tree, where nodes at level $i$ are the outcomes of the $(i-1)$th step, and each node at level $i$ will emanate as many nodes as there are possible outcomes for the $i$th step at that node. Every edge emanating out of a node at level $i$ denotes the conditional event of the outcome at the tail of edge happens given that we are node at the head of the edge. 
\\
A unidirectional path from the top of the tree to one of its leaves will describe one outcome for the experiment. The sample space consists of all possible unidirectional paths from the top of the tree to it's leaves. 
\\
The multiplication principle can be used to assign probabilities to all simple events. 
\begin{ex}
    We revisit the experiment where we want to assign probabilities to the simple events associated to tossing a coin two times. 
\begin{enumerate}
\item Consider the experiment that we toss a fair coin two times. \\

\begin{tikzpicture}[>=triangle 60,every node/.style={anchor=west}]
\matrix (m) [matrix of nodes, row sep=0em, column sep=3em]
{You\\};
\draw (m-1-1.east) --+ (2.5,0) node[right] (h) {$H_1$};
\draw (m-1-1.east) --+ (2.5,-2) node[right] (t) {$T_1$};
\draw (h.east) --+ (2.5,0) node[right] (hh) {$H_2$};
\draw (h.east) --+ (2.5,-1) node[right] (ht) {$T_2$};
\draw (t.east) --+ (2.5,0) node[right] (th) {$H_2$};
\draw (t.east) --+ (2.5,-1) node[right] (tt) {$T_2$};
\dr{tt}{TT}{$P(TT)=1/4$}{2.5}{0}
\dr{th}{TH}{$P(TH)=1/4$}{2.5}{0}
\dr{ht}{HT}{$P(HT)=1/4$}{2.5}{0}
\dr{hh}{HH}{$P(HH)=1/4$}{2.5}{0}         

\node[xshift=3.5cm,text width=2.5cm,above=.5cm] at (m-1-1.north east) {\bf step 1};
\node[xshift=7cm,text width=2.5cm,above=.5cm] at (m-1-1.north east) {\bf step 2};
\node[xshift=11cm,text width=2.5cm,above=.5cm] at (m-1-1.north east) {$P(E)$};

\draw (m-1-1.east) edge["1/2"] (h.west);
\draw (m-1-1.east) edge node[below]{1/2} (t.west);
\draw (h.east) edge["1/2"] (hh.west);
\draw (h.east) edge node[below]{1/2} (ht.west);
\draw (t.east) edge["1/2"] (th.west);
\draw (t.east) edge node[below]{1/2} (tt.west) ;
\end{tikzpicture}\\

\qquad\qquad
\begin{tabular}{|c|c|c|c|c|}
\hline
 \cellcolor{lightgray} Event $E$ & $HH$ & $HT$ & $TH$ & $TT$\\
\hline
 \cellcolor{lightgray} $P(E)$ & $1/4$ & $1/4$ & $1/4$ &$1/4$\\
\hline 
\end{tabular}\\

\item Now consider first we flip a fair coin, then an unfair coin. \\
The second coin has probability $P(H_2) = 1/4$ and $P(T_2) = 3/4$. 

\begin{tikzpicture}[>=triangle 60,every node/.style={anchor=west}]
\matrix (m) [matrix of nodes, row sep=0em, column sep=3em]
{You\\};
\draw (m-1-1.east) --+ (2.5,0) node[right] (h) {$H_1$};
\draw (m-1-1.east) --+ (2.5,-2) node[right] (t) {$T_1$};
\draw (h.east) --+ (2.5,0) node[right] (hh) {$H_2$};
\draw (h.east) --+ (2.5,-1) node[right] (ht) {$T_2$};
\draw (t.east) --+ (2.5,0) node[right] (th) {$H_2$};
\draw (t.east) --+ (2.5,-1) node[right] (tt) {$T_2$};
\dr{tt}{TT}{$P(TT)=3/8$}{2.5}{0}
\dr{th}{TH}{$P(TH)=1/8$}{2.5}{0}
\dr{ht}{HT}{$P(HT)=3/8$}{2.5}{0}
\dr{hh}{HH}{$P(HH)=1/8$}{2.5}{0}         

\node[xshift=3.5cm,text width=2.5cm,above=.5cm] at (m-1-1.north east) {\bf step 1};
\node[xshift=7cm,text width=2.5cm,above=.5cm] at (m-1-1.north east) {\bf step 2};
\node[xshift=11cm,text width=2.5cm,above=.5cm] at (m-1-1.north east) {$P(E)$};

\draw (m-1-1.east) edge["1/2"] (h.west);
\draw (m-1-1.east) edge node[below]{1/2} (t.west);
\draw (h.east) edge["1/4"] (hh.west);
\draw (h.east) edge node[below]{3/4} (ht.west);
\draw (t.east) edge["1/4"] (th.west);
\draw (t.east) edge node[below]{3/4} (tt.west) ;
\end{tikzpicture}\\

\qquad\qquad
\begin{tabular}{|c|c|c|c|c|}
\hline
 \cellcolor{lightgray} Event $E$ & $HH$ & $HT$ & $TH$ & $TT$\\
\hline
 \cellcolor{lightgray} $P(E)$ & $1/8$ & $3/8$ & $1/8$ &$3/8$\\
\hline 
\end{tabular}\\



\item Now consider first we flip an unfair coin, then a fair coin. \\
The first coin has probability $P(H_1) = 1/4$ and $P(T_1) = 3/4$. 

\begin{tikzpicture}[>=triangle 60,every node/.style={anchor=west}]
\matrix (m) [matrix of nodes, row sep=0em, column sep=3em]
{You\\};
\draw (m-1-1.east) --+ (2.5,0) node[right] (h) {$H_1$};
\draw (m-1-1.east) --+ (2.5,-2) node[right] (t) {$T_1$};
\draw (h.east) --+ (2.5,0) node[right] (hh) {$H_2$};
\draw (h.east) --+ (2.5,-1) node[right] (ht) {$T_2$};
\draw (t.east) --+ (2.5,0) node[right] (th) {$H_2$};
\draw (t.east) --+ (2.5,-1) node[right] (tt) {$T_2$};
\dr{tt}{TT}{$P(TT)=3/8$}{2.5}{0}
\dr{th}{TH}{$P(TH)=3/8$}{2.5}{0}
\dr{ht}{HT}{$P(HT)=1/8$}{2.5}{0}
\dr{hh}{HH}{$P(HH)=1/8$}{2.5}{0}         

\node[xshift=3.5cm,text width=2.5cm,above=.5cm] at (m-1-1.north east) {\bf step 1};
\node[xshift=7cm,text width=2.5cm,above=.5cm] at (m-1-1.north east) {\bf step 2};
\node[xshift=11cm,text width=2.5cm,above=.5cm] at (m-1-1.north east) {$P(E)$};

\draw (m-1-1.east) edge["1/4"] (h.west);
\draw (m-1-1.east) edge node[below]{3/4} (t.west);
\draw (h.east) edge["1/2"] (hh.west);
\draw (h.east) edge node[below]{1/2} (ht.west);
\draw (t.east) edge["1/2"] (th.west);
\draw (t.east) edge node[below]{1/2} (tt.west) ;
\end{tikzpicture}\\

\qquad\qquad
\begin{tabular}{|c|c|c|c|c|}
\hline
 \cellcolor{lightgray} Event $E$ & $HH$ & $HT$ & $TH$ & $TT$\\
\hline
 \cellcolor{lightgray} $P(E)$ & $1/8$ & $1/8$ & $3/8$ &$3/8$\\
\hline 
\end{tabular}\\

\item One might ask if it is possible to construct the following distribution table:\\

\qquad\qquad
\begin{tabular}{|c|c|c|c|c|}
\hline
 \cellcolor{lightgray} Event $E$ & $HH$ & $HT$ & $TH$ & $TT$\\
\hline
 \cellcolor{lightgray} $P(E)$ & $1/8$ & $1/8$ & $4/8$ &$2/8$\\
\hline 
\end{tabular}\\

To construct such a distribution table, we consider the following experiment: In step 1, we toss a coin with $P(H_1) = 1/4$; In step 2, if we got $H_1$ in step 1, then we toss a coin with $P(H_2) = 1/2$, if we got $T_1$ in step 1, then we toss a coin with $P(H_2) = 4/6$.


\begin{tikzpicture}[>=triangle 60,every node/.style={anchor=west}]
\matrix (m) [matrix of nodes, row sep=0em, column sep=3em]
{You\\};
\draw (m-1-1.east) --+ (2.5,0) node[right] (h) {$H_1$};
\draw (m-1-1.east) --+ (2.5,-2) node[right] (t) {$T_1$};
\draw (h.east) --+ (2.5,0) node[right] (hh) {$H_2$};
\draw (h.east) --+ (2.5,-1) node[right] (ht) {$T_2$};
\draw (t.east) --+ (2.5,0) node[right] (th) {$H_2$};
\draw (t.east) --+ (2.5,-1) node[right] (tt) {$T_2$};
\dr{hh}{HH}{$P(HH)=1/8$}{2.5}{0}         
\dr{ht}{HT}{$P(HT)=1/8$}{2.5}{0}
\dr{th}{TH}{$P(TH)=4/8$}{2.5}{0}
\dr{tt}{TT}{$P(TT)=2/8$}{2.5}{0}

\node[xshift=3.5cm,text width=2.5cm,above=.5cm] at (m-1-1.north east) {\bf step 1};
\node[xshift=7cm,text width=2.5cm,above=.5cm] at (m-1-1.north east) {\bf step 2};
\node[xshift=11cm,text width=2.5cm,above=.5cm] at (m-1-1.north east) {$P(E)$};

\draw (m-1-1.east) edge["1/4"] (h.west);
\draw (m-1-1.east) edge node[below]{3/4} (t.west);
\draw (h.east) edge["1/2"] (hh.west);
\draw (h.east) edge node[below]{1/2} (ht.west);
\draw (t.east) edge["4/6"] (th.west);
\draw (t.east) edge node[below]{2/6} (tt.west) ;
\end{tikzpicture}\\

We see that this experiment gives the desired distribution table.

\item Now suppose we toss a two-headed coin two times, that is, $P(H_1) = P(H_2) = 1$.

\begin{tikzpicture}[>=triangle 60,every node/.style={anchor=west}]
\matrix (m) [matrix of nodes, row sep=0em, column sep=3em]
{You\\};
\draw (m-1-1.east) --+ (2.5,0) node[right] (h) {$H_1$};
\draw (m-1-1.east) --+ (2.5,-2) node[right] (t) {$T_1$};
\draw (h.east) --+ (2.5,0) node[right] (hh) {$H_2$};
\draw (h.east) --+ (2.5,-1) node[right] (ht) {$T_2$};
\draw (t.east) --+ (2.5,0) node[right] (th) {$H_2$};
\draw (t.east) --+ (2.5,-1) node[right] (tt) {$T_2$};
\dr{tt}{TT}{$P(TT)=0$}{2.5}{0}
\dr{th}{TH}{$P(TH)=0$}{2.5}{0}
\dr{ht}{HT}{$P(HT)=0$}{2.5}{0}
\dr{hh}{HH}{$P(HH)=1$}{2.5}{0}         

\node[xshift=3.5cm,text width=2.5cm,above=.5cm] at (m-1-1.north east) {\bf step 1};
\node[xshift=7cm,text width=2.5cm,above=.5cm] at (m-1-1.north east) {\bf step 2};
\node[xshift=11cm,text width=2.5cm,above=.5cm] at (m-1-1.north east) {$P(E)$};

\draw (m-1-1.east) edge["1"] (h.west);
\draw (m-1-1.east) edge node[below]{0} (t.west);
\draw (h.east) edge["1"] (hh.west);
\draw (h.east) edge node[below]{0} (ht.west);
\draw (t.east) edge["1"] (th.west);
\draw (t.east) edge node[below]{0} (tt.west) ;
\end{tikzpicture}\\

\qquad\qquad
\begin{tabular}{|c|c|c|c|c|}
\hline
 \cellcolor{lightgray} Event $E$ & $HH$ & $HT$ & $TH$ & $TT$\\
\hline
 \cellcolor{lightgray} $P(E)$ & $1$ & $0$ & $0$ & $0$\\
\hline 
\end{tabular}\\
\end{enumerate}
\hfill\break
\hfill\break

Observe that in the coin toss examples above, we can consider the the following two tree diagram\\
\begin{center}
\begin{tikzpicture}[>=triangle 60,every node/.style={anchor=west}]
\matrix (m) [matrix of nodes, row sep=0em, column sep=3em]
{Experimentor\\};
\draw (m-1-1.east) --+ (1,0.7) node[right] (h) {$H_1$};
\draw (m-1-1.east) --+ (1,-0.7) node[right] (t) {$T_1$};
\draw (h.east) --+ (1,1) node[right] (hh) {$H_2$};
\draw (h.east) --+ (1,0) node[right] (ht) {$T_2$};
\draw (t.east) --+ (1.05,0) node[right] (th) {$H_2$};
\draw (t.east) --+ (1.05,-1) node[right] (tt) {$T_2$};
\dr{tt}{TT}{$T_1T_2$}{1}{0}
\dr{th}{TH}{$T_1H_2$}{1}{0}
\dr{ht}{HT}{$H_1T_2$}{1}{0}
\dr{hh}{HH}{$H_1H_2$}{1}{0}
\dr{HH}{ohh}{$H_1$}{1}{0}
\dr{TH}{oth}{$T_1$}{1.15}{1.4}
\dr{HT}{oht}{$H_1$}{1.15}{-1.4}         
\dr{TT}{ott}{$T_1$}{1.4}{0}
\dr{oth}{oh}{}{1.1}{0}         
\dr{ohh}{oh}{$H_2$}{1}{-1}
\dr{ott}{ot}{}{1}{1}
\dr{oht}{ot}{$T_2$}{1}{0}         
\dr{ot}{obs}{Observer}{1.1}{0.7}
\dr{oh}{obs}{}{1}{-0.7}

\node[xshift=5.3cm,text width=2.5cm,above=1.9cm] at (m-1-1.north east) {sample space};

\draw (m-1-1.east) edge node[above, rotate=45]{\tiny $P(H_1)$ \normalsize} (h.west);
\draw (m-1-1.east) edge node[below, rotate=-39]{\tiny $P(T_1)$ \normalsize} (t.west);
\draw (h.east) edge node[above, rotate=50]{\tiny $P(H_2|H_1)$ \normalsize} (hh.west);
\draw (h.east) edge node[below, rotate=0]{\tiny $P(T_2|H_1)$ \normalsize} (ht.west);
\draw (t.east) edge node[above, rotate=0]{\tiny $P(H_2|T_1)$ \normalsize} (th.west);
\draw (t.east) edge node[below, rotate=-45]{\tiny $P(T_2|T_1)$} (tt.west) ;
\draw (ohh.east) edge node[above, rotate=-45]{\tiny $P(H_1|H_2)$} (oh.west) ;
\draw (oth.east) edge node[below, rotate=0]{\tiny $P(T_1|H_2)$} (oh.west) ;
\draw (oht.east) edge node[above, rotate=0]{\tiny $P(H_1|T_2)$} (ot.west) ;
\draw (ott.east) edge node[below, rotate=45]{\tiny $P(T_1|T_2)$} (ot.west) ;
\draw (ot.east) edge node[below, rotate=45]{\tiny $P(T_2)$} (obs.west) ;
\draw (oh.east) edge node[above, rotate=-39]{\tiny $P(H_2)$} (obs.west) ;

\draw[-latex] (0,-2.5) -- ++ (5cm,0) node[midway,below,align=center]{Experiment arrow of time};
\draw[-latex] (15,-2.5) -- ++ (-5cm,0) node[midway,below,align=center]{Observer arrow of time};
\end{tikzpicture}
\end{center}
The connecting piece between these two trees is the law of total probability and Bayes' theorem.
\end{ex}

\section{Bayes' Theorem}
We are now ready to state the general theorems  
\begin{thm}[Law of Total Probability]
Suppose $\{A_1, A_2, \dots, A_k\}$ is a partition for the sample space and $B$ is any event, then
\begin{eqnarray*}
    P(B) &=& \sum_{j=1}^k P(B\cap A_j)\\
    &=&\sum_{j=1}^k P(A_j) P(B \mid A_j)
\end{eqnarray*}
\end{thm}
The last equation in the theorem is just the application of the multiplication principle. 


\begin{thm}[Bayes' Theorem]
    Suppose $\{A_1, A_2, \dots, A_k\}$ is a partition for the sample space and $B$ is any event, then
\begin{eqnarray*}
    P(A_i \mid B) = \frac{P(A_i)P(B\mid A_i)}{\sum_{j=1}^k P(A_j) P(B \mid A_j)}
\end{eqnarray*}
\end{thm}
\begin{proof}
Using the fact that $\{A_1, A_2, \dots, A_k\}$ is a partition and the law of total probability we have:
    \begin{eqnarray*}
    P(A_i \mid B) &=& \frac{P(A_i \cap B)}{P(B)}\\
    &=& \frac{P(A_i \cap B) }{\sum_{j=1}^k P(B\cap A_j)}\\
    &=&\frac{P(A_i)P(B\mid A_i)}{\sum_{j=1}^k P(A_j) P(B \mid A_j)}
\end{eqnarray*}
\end{proof}


\begin{ex} %% Choosing a coin and tossing it two times
A bag contains two coins, one is a fair coin denoted as $F$, and the other is a two headed coin denoted as $U$. \\
We randomly choose a coin from the bag and toss it once, this can be described by the following tree diagram.\\
\begin{center}
\begin{tikzpicture}[grow=right, sloped]
\node[bag] {}
    child {
        node[bag] {U}        
            child {
                node[end, label=right:
                    {\text{H}}] {}
                edge from parent
                node[above]  {$P(H\mid U)=1$}
                node[below] {\begin{scriptsize}\text{Note }$P(T\mid U)=0$ \end{scriptsize} }
            }
            edge from parent 
            node[above] {$P(U)=\frac{1}{2}$}
    }
    child {
        node[bag] {F}        
        child {
                node[end, label=right:
                    {\text{T}}] {}
                edge from parent
                node[above] {$P(T\mid F)=\frac{1}{2}$}
            }
            child {
                node[end, label=right:
                    {\text{H}}] {}
                edge from parent
                nnode[above] {$P(T\mid F)=\frac{1}{2}$}
            }
        edge from parent         
            node[above] {$P(F)=\frac{1}{2}$}
    };
\end{tikzpicture}
\end{center}
The sample space $\mathcal{S}=\{FH,FT,UH\}$, which is $F\cap H$, $F\cap T,U\cap H$ respectively.\\
\hfill\\
We can calculate the probability that the tossed coin lands heads as follows \begin{align*}
P(H)&=P(F\cap H)+P(U\cap H)\\
&=P(H\mid F)\cdot P(F)+P(H\mid U)\cdot P(U)\\
&=\frac{1}{2}\cdot \frac{1}{2}+1\cdot \frac{1}{2}=\frac{1}{4}+\frac{1}{2}=\frac{3}{4}
\end{align*}
Observing that $P(H\mid F)=\frac{1}{2}\neq \frac{3}{4}=P(H)$, helps us conclude that observing a $H$ on the coin toss and choosing the fair coin $F$ are two events that are not \textbf{not} independent of each other. 
\\

We can also calculate the conditional probability
\begin{align*}
P(F|H)&=\frac{P(F\cap H)}{P(H)}\\
&=\frac{P(H\mid F)\cdot P(F)}{P(H)}\\
&=\frac{\frac{1}{2}\cdot \frac{1}{2}}{\frac{3}{4}}=\frac{1}{4}\cdot \frac{4}{3}=\frac{1}{3}
\end{align*}
Therefore, $P(F)=\frac{1}{2}>\frac{1}{3}=P(F|H)$. This makes intuitive sense, as $P(F) = P(\text{tossed coin is a fair coin})$ will be lowered after we observe the $H$ outcome.
\end{ex}


\begin{ex}
Suppose we now randomly choose a coin from the bag containing a fair coin and a two headed coin, and toss it two times independently. 
This experiment can be visualized in the following tree diagram:
\begin{center}
\begin{tikzpicture}[grow=right, sloped]
\node[bag] {}
    child {
        node[bag] {U}        
            child {
                node[bag]{H}
                child{
                 node[end, label=right:
                    {\text{H}}] {}
                edge from parent
                node[above]  {\begin{scriptsize}$P(H\mid H\cap U)=1$\end{scriptsize}}
                }
                edge from parent
                node[above]  {$P(H\mid U)=1$}
                node[below] {\begin{scriptsize}\text{Note }$P(T\mid U)=0$ \end{scriptsize} }
            }
            edge from parent 
            node[above] {\begin{small}$P(U)=\frac{1}{2}$\end{small}}
    }
    child {
        node[bag] {F}        
        child {
                node[bag] {T}
                child{
                 node[end, label=right:
                    {\text{T}}] {}
                edge from parent
                node[above]  {\begin{scriptsize}$P(T\mid T\cap F)=\frac{1}{2}$\end{scriptsize}}
                }
                child{
                 node[end, label=right:
                    {\text{H}}] {}
                edge from parent
                node[above]  {\begin{scriptsize}$P(H\mid T\cap F)=\frac{1}{2}$\end{scriptsize}}
                }
                edge from parent
                node[above] {\begin{small}$P(T\mid F)=\frac{1}{2}$\end{small}}
            }
            child {
                node[bag] {H}
                child{
                 node[end, label=right:
                    {\text{H}}] {}
                edge from parent
                node[above]  {\begin{scriptsize}$P(H\mid H\cap F)=\frac{1}{2}$\end{scriptsize}}
                }
                child{
                 node[end, label=right:
                    {\text{T}}] {}
                edge from parent
                node[above]  {\begin{scriptsize}$P(T\mid H\cap F)=\frac{1}{2}$\end{scriptsize}}
                }
                edge from parent
                node[above] {\begin{small}$P(T\mid F)=\frac{1}{2}$\end{small}}
            }
        edge from parent         
            node[above] {$P(F)=\frac{1}{2}$}
    };
\end{tikzpicture}
\end{center}
The sample space is $\mathcal{S}=\{FHT,FHH,FTH,FTT,UHH\}$. 

We can calculate the probabilities of each of the simple events by taking products along the path to the given outcome, for example:
\begin{eqnarray*}
    P(HHF)&=& P(H\cap (H\cap F))\\
        &=&P(H\mid H\cap F)\cdot P(H\cap F)\\
        &=&P(H\mid H\cap F)\cdot P(H\mid F)\cdot P(F)\\
        &=&\text{Product along path to $HHF$}
\end{eqnarray*}

Suppose we want to calculate $P(HH) = P(\text{the tossed coin lands heads both times})$, we do it as follows:
\begin{eqnarray*}
P(HH)&=&P(FHH)+P(UHH)\\
&=&P(H\mid HF)\cdot P(HF)+P(H\mid HU)\cdot P(HU)\\
&=&\frac{1}{2}\cdot \frac{1}{2}\cdot \frac{1}{2}+\frac{1}{2}\cdot 1\cdot 1\\
&=&\frac{1}{8}+\frac{1}{2}=\frac{5}{8}
\end{eqnarray*}

We can also calculate the conditional probability 
\begin{eqnarray*}
    P(HH\mid F)&=& P(\text{observed two consecutive heads given that a fair coin was tossed}\\
    &=&\frac{P(HHF)}{P(F)}\\
    &=&\frac{\frac{1}{2}\cdot \frac{1}{2}\cdot \frac{1}{2}}{\frac{1}{2}}\\
    &=&\frac{1}{4}
\end{eqnarray*}

Therefore, $P(HH)=\frac{5}{8}>\frac{1}{4}=P(HH\mid F)$, the information that we tossed a fair coin affected the probability of getting two consecutive heads - this information results in a lower probability. 

We could also consider the probability  
\begin{eqnarray*}
P(F\mid HH) &=& P(\text{we tossed a fair coin given that $HH$ was observed})\\
&=& \frac{P(FHH)}{P(HH)}=\frac{P(HH\mid F)\cdot P(F)}{P(HH)}\\
&=&\frac{\frac{1}{4}\cdot \frac{1}{2}}{\frac{5}{8}}=\frac{1}{8}\cdot \frac{8}{5}\\
&=&\frac{1}{5}
\end{eqnarray*}

Therefore, $P(F\mid HH)=\frac{1}{5}<\frac{1}{3}=P(F\mid H)<\frac{1}{2}=P(F)$, which aligns with the intuition that as more heads are observed, our confidence that a fair coin was tossed must decrease.
\end{ex}

\begin{ex}[Disease Testing]
Suppose we consider a disease in a certain population, and 
$$ D:= \{\text{randomly chosen individual has the disease}\}$$ 
so that 
$$D^c:= \{\text{randomly chosen individual does  not have the disease}\}$$
Observe that $$D\cup D^c := \{ \text{entire population}\} = \mathcal{S}$$

Additionally, we are given a test, which when administered will determine if the person has the disease or not. Suppose 
$$ \text{Pos} := \{\text{test result is positive, i.e. test claims you have the disease}\}$$

$$ \text{Neg} := \{\text{test result is negative, i.e. test claims you do not have the disease}\}$$
\\
If everyone in the population was administered the test, we can observe that $$\text{Pos}\cup \text{Neg} := \{ \text{entire population}\} = \mathcal{S}$$

We now consider the following probabilities:
\begin{eqnarray*}
    P(D)&:=& P(\text{a randomly chosen individual from the population has the disease}) \\
    &:=& p \in (0,1) 
\end{eqnarray*}
this is sometimes called the prevalence of the disease. Additionally, we are given 
\begin{eqnarray*}
    P(\text{test correctly identifies the disease})&:=& q \in (0,1) 
\end{eqnarray*}

This is sometimes called the accuracy of the test. Using this we can calculate the following conditional probabilities:
\begin{eqnarray*}
    P(\text{Pos}\mid D)&=& P(\text{Neg}\mid D^c) = q \\
    P(\text{Pos} \mid D^c) &=& P(\text{Neg}\mid D) = 1-q
\end{eqnarray*}

Depending on the information that is available, we can consider two differently ordered experiments. 
\\

\begin{eqnarray*}
    \text{Step 1} &:=& \text{Check if you have the disease} \\
    \text{Step 2} &:=& \text{What is the test result?}
\end{eqnarray*} 

The tree that we get in this sequence is called the "Experimenter's Tree", as the experimenter testing the accuracy of the test needs to know the disease-status of the individual on whom the test is administered. This tree can be visualized as follows:

\qquad\begin{tikzpicture}[>=triangle 60,every node/.style={anchor=west}]
\matrix (m) [matrix of nodes, row sep=0em, column sep=3em]
{Experimenter\\};
\draw (m-1-1.east) --+ (1,0.7) node[right] (h) {$D$};
\draw (m-1-1.east) --+ (1,-0.7) node[right] (t) {$D^c$};
\draw (h.east) --+ (1.1,1) node[right] (hh) {$ \text{Pos} $};
\draw (h.east) --+ (1.1,0) node[right] (ht) {$ \text{Neg} $};
\draw (t.east) --+ (1,0) node[right] (th) {$ \text{Pos} $};
\draw (t.east) --+ (1,-1) node[right] (tt) {$ \text{Neg} $};
\dr{tt}{TT}{$(D^c,\, \text{Neg} )$}{1}{0}
\dr{th}{TH}{$(D^c,\, \text{Pos} )$}{1}{0}
\dr{ht}{HT}{$(D,\, \text{Neg} )$}{1}{0}
\dr{hh}{HH}{$(D,\, \text{Pos} )$}{1}{0}

\node[xshift=5.3cm,text width=2.5cm,above=1.9cm] at (m-1-1.north east) {sample space};

\draw (m-1-1.east) edge node[above, rotate=45]{\tiny $p$ \normalsize} (h.west);
\draw (m-1-1.east) edge node[below, rotate=-39]{\tiny $1-p$ \normalsize} (t.west);
\draw (h.east) edge node[above, rotate=50]{\tiny $q$ \normalsize} (hh.west);
\draw (h.east) edge node[below, rotate=0]{\tiny $1-q$ \normalsize} (ht.west);
\draw (t.east) edge node[above, rotate=0]{\tiny $1-q$ \normalsize} (th.west);
\draw (t.east) edge node[below, rotate=-45]{\tiny $q$} (tt.west) ;
\end{tikzpicture}


From the user's (of the test) point of view, we have the following steps
\begin{eqnarray*}
    \text{Step 1} &:=& \text{administer the test} \\
    \text{Step 2} &:=& \text{Do I have the disease?}
\end{eqnarray*} 
We can see that the user's order of operation is exactly reverse of the experimenter. 
\\
The user of the test is aware of the test result, and based on that information wants to know their disease status. 
\\

We can put all of this information in the following two tree diagrams. 


%%%%%%%%%%%%%%%%%%%%%%%%%%%%%%% TREE DIAGRAMS %%%%%%%%%%%%%%%%%%%%%%
\begin{center}
\begin{tikzpicture}[>=triangle 60,every node/.style={anchor=west}]
\matrix (m) [matrix of nodes, row sep=0em, column sep=3em]
{Experimenter\\};
\draw (m-1-1.east) --+ (1,0.7) node[right] (h) {$D$};
\draw (m-1-1.east) --+ (1,-0.7) node[right] (t) {$D^c$};
\draw (h.east) --+ (1.1,1) node[right] (hh) {$ \text{Pos} $};
\draw (h.east) --+ (1.1,0) node[right] (ht) {$ \text{Neg} $};
\draw (t.east) --+ (1,0) node[right] (th) {$ \text{Pos} $};
\draw (t.east) --+ (1,-1) node[right] (tt) {$ \text{Neg} $};
\dr{tt}{TT}{$(D^c,\, \text{Neg} )$}{1}{0}
\dr{th}{TH}{$(D^c,\, \text{Pos} )$}{1}{0}
\dr{ht}{HT}{$(D,\, \text{Neg} )$}{1}{0}
\dr{hh}{HH}{$(D,\, \text{Pos} )$}{1}{0}
\dr{HH}{ohh}{$D$}{1}{0}
\dr{TH}{oth}{$D^c$}{1}{1.4}
\dr{HT}{oht}{$D$}{1}{-1.4}         
\dr{TT}{ott}{$D^c$}{1}{0}
\dr{oth}{oh}{}{1}{0}         
\dr{ohh}{oh}{$ \text{Pos} $}{1.5}{-1}
\dr{ott}{ot}{}{1}{1}
\dr{oht}{ot}{$ \text{Neg} $}{1.35}{0}         
\dr{ot}{obs}{User}{1}{0.7}
\dr{oh}{obs}{}{1}{-0.7}

\node[xshift=5.3cm,text width=2.5cm,above=1.9cm] at (m-1-1.north east) {sample space};

\draw (m-1-1.east) edge node[above, rotate=45]{\tiny $p$ \normalsize} (h.west);
\draw (m-1-1.east) edge node[below, rotate=-39]{\tiny $1-p$ \normalsize} (t.west);
\draw (h.east) edge node[above, rotate=50]{\tiny $q$ \normalsize} (hh.west);
\draw (h.east) edge node[below, rotate=0]{\tiny $1-q$ \normalsize} (ht.west);
\draw (t.east) edge node[above, rotate=0]{\tiny $1-q$ \normalsize} (th.west);
\draw (t.east) edge node[below, rotate=-45]{\tiny $q$} (tt.west) ;
\draw (ohh.east) edge node[above, rotate=-35]{\tiny $P(D| \text{Pos} )$} (oh.west) ;
\draw (oth.east) edge node[below, rotate=0]{\tiny $P(D^c| \text{Pos} )$} (oh.west) ;
\draw (oht.east) edge node[above, rotate=0]{\tiny $P(D| \text{Neg} )$} (ot.west) ;
\draw (ott.east) edge node[below, rotate=45]{\tiny $P(D^c| \text{Neg} )$} (ot.west) ;
\draw (ot.east) edge node[below, rotate=45]{\tiny $P( \text{Neg} )$} (obs.west) ;
\draw (oh.east) edge node[above, rotate=-35]{\tiny $P( \text{Pos} )$} (obs.west) ;

\draw[-latex] (0,-2.5) -- ++ (5cm,0) node[midway,below,align=center]{Experimenter's arrow of time};
\draw[-latex] (15,-2.5) -- ++ (-5cm,0) node[midway,below,align=center]{User's arrow of time};
\end{tikzpicture}
\end{center}

In this setting, the accuracy of the test relates to $P(\text{Pos}|D^c)$ and $P(\text{Neg}|D)$ being small, as they correspond to the probabilities of incorrectly identifying the disease using the test. On the other hand, $P(\text{Pos}|D)$ and $P(\text{Neg}|D^c)$ should be large as they correspond to the probabilities of correctly identifying the disease using the test. 
\\

For the observer, one expects $P(D|\text{Pos})$ and $P(D^c| \text{Neg})$ to be large, and $P(D^c|\text{Pos})$ and $P(D|\text{Neg})$ to be small.
\\

Observe that 
\begin{enumerate}
    \item The trees for the experimenter and the observer result in the same sample space.
    \item Using the Law of Total Probability and the Bayes' theorem, we can calculate all the probabilities in the user's tree if the all the probabilities in the experimenter's tree are known. 
\end{enumerate}

We calculate as follows:

\begin{align*}
P(\text{Pos}) &= P(D, \text{Pos})+P(D^c, \text{Pos}) = p \cdot q + (1-p)\cdot (1-q)\,,
\end{align*}
Therefore, 
\begin{align*}
P(D|\text{Pos}) &= \frac{P(D, \text{Pos})}{P(\text{Pos})} = \frac{pq}{pq+(1-p)(1-q)}\,, 
\end{align*}
\begin{align*}
P(D^c| \text{Pos}) = \frac{P(D^c, \text{Pos})}{P(\text{Pos})} = \frac{(1-p)(1-q)}{pq+(1-p)(1-q)}\,.
\end{align*}
Similarly, we can calculate
\begin{align*}
P(\text{Neg}) = p\cdot(1-q) + q\cdot (1-p)\,, 
\end{align*}
\begin{align*}
P(D^c|\text{Neg}) = \frac{q(1-p)}{q(1-p)+p(1-q)}\,,\qquad
P(D|\text{Neg}) = \frac{p(1-q)}{q(1-p) +p(1-q)}\,.
\end{align*}
This finishes the calculation of all the probabilities in the user's tree. 
\\

As mentioned before, we would like to have 
\begin{align*}
P(D|\text{Pos}) = \frac{pq}{pq+(1-p)(1-q)}\,,\qquad
\text{and}\qquad P(D^c|\text{Neg}) = \frac{(1-p)q}{(1-p)q+p(1-q)}
\end{align*}
to be large.
\\

We can consider two special cases:
\begin{enumerate}
\item Suppose the disease is highly prevalent, for instance, $P(D) = p = 0.9999$. That is, 9999 out of 10000 people on average have the disease. Furthermore, let $P(\text{test correctly identifies the disease})= q = 0.9999$. That is, the test is on average correct 9999 times out of 10000 times. For the test to be useful, we want to have both $P(D|\text{Pos})$ and $P(D^c|\text{Neg})$ to be large. In this case, we compute their numerical values
\begin{align*}
P(D|\text{Pos}) = \frac{pq}{pq+(1-p)(1-q)}  = \frac{0.9999 \cdot 0.9999}{0.9999\cdot 0.9999 + 0.0001 \cdot 0.0001}\approx 1\,.
\end{align*}
However, we see that
\begin{align*}
P(D^c|\text{Neg}) = \frac{(1-p)q}{(1-p)q+p(1-q)} = \frac{0.0001 \cdot 0.9999}{0.0001\cdot 0.9999 + 0.9999\cdot 0.0001} = 0.5\,,
\end{align*}
from which we see that a negative test result will correctly identify the test with probability 0.5, perhaps the user is better off tossing a fair coin (instead of buying the test). The test is not very useful in this sense. 

\item On the other hand, if we have a rare disease, say for instance $P(D) = p = 0.0001$. Suppose that the test accuracy is  $P(\text{test correctly identifies disease}) = 0.9999=q$. For the test to be useful, we again want to have both $P(D|\text{Pos})$ and $P(D^c|\text{Neg})$ to be large, but we again see that 
\begin{align*}
P(D|\text{Pos}) = \frac{pq}{pq+(1-p)(1-q)} = \frac{0.0001\cdot 0.9999}{0.0001 \cdot 0.9999 + 0.9999 \cdot 0.0001} = 0.5\,,
\end{align*}
which is again equivalent to tossing a fair coin toss. 
\end{enumerate}


\end{ex}




\section[Probability and Counting]{Probability and Counting}
Suppose we are in the setting where the sample space $\mathcal{S}$ satisfies the following two conditions:
\begin{enumerate}
\item $\mathcal{S}$ is a finite set. That is, $\mathcal{S}$ is the sample space of an experiment with finitely many outcomes.
\item Every outcome in $\mathcal{S}$ is equally likely. That is, if $\mathcal{S}=\{s1,s_2,\cdots,s_n\}$ then $P(\{s_i\})=\frac{1}{n}$ for all $i=1,2,3,\cdots,n$
\end{enumerate}
Let $A\subset \mathcal{S}$ be an event. Since every outcome in $\mathcal{S}$ is equally likely 
\begin{eqnarray*}
P(A)&=&\text{Probability that A occurs}\\
&=& \sum_{s\in A} P(\{s\})\\
&=& \sum_{s\in A} \frac{1}{n}\\
&=& \frac{n(A)}{n(S)}
\end{eqnarray*}

where $n(A)$ is the number of objects in $A$. 
\\
Therefore, when the sample space is finite and all outcomes in the sample space are equally likely, calculating the probability of an event $A$ is same as counting outcomes in $A$. In this scenario, we only need to be able to count! 
\\

We state the first important theorem of counting. 
\begin{thm}[Fundamental Theorem of Counting]
 Suppose there are $k$ tasks: $T_1,T_2,\cdots,T_k$ that can be performed in $n_1,n_2,\cdot,n_k$ ways respectively. Let $T$ be the task of performing $T_1,T_2,\cdot,T_k$ sequentially. Then the total number of ways to perform the task $T$ is 
 $$n_1\times n_2 \times \cdots \times n_k$$ 
\end{thm} 

\begin{proof}
    We prove this theorem using the principle of mathematical induction applied to the number of tasks. We can structure the task $T$ as a tree, with level $i$ in the tree corresponding to the $i$th task $T_i$. 
\end{proof}

\begin{ex}
    Add examples of counting. 
\end{ex}

\subsection{Choosing from $n$ distinct objects}
We will often want to count the number of ways of selecting $k$ objects from a set of $n$ distinct objects.
\begin{ex}
Suppose we want to count the total number of ways one can choose 4 digits from the set of 10 digits $\{0,1,2,3,4,5,6,7,8,9\}$.
\\

The answer to this we need to understand how the digits are selected. Do we place the digit back into the set before the subsequent choice (with/without replacement)? Does the order in which the 4 digits are selected matter?
This is leads to the following 4 possibilities:

\end{ex}\begin{center}
\begin{tabular}{l|l|l|}
\cline{2-3}
                                                                                       & Without Replacement                                                                                    & With Replacement                                                                                   \\ \hline
\multicolumn{1}{|l|}{Ordered}                                                    & \begin{tabular}[c]{@{}l@{}}(1,2,4,5) different from (1,5,2,4)\\ (1,1,2,5) is not possible\end{tabular} & \begin{tabular}[c]{@{}l@{}}(1,2,4,5) different from (1,5,2,4)\\ (1,1,2,5) is possible\end{tabular} \\ \hline
\multicolumn{1}{|l|}{\begin{tabular}[c]{@{}l@{}}Unordered\end{tabular}} & \begin{tabular}[c]{@{}l@{}}(1,2,3,4) is same as (4,3,2,1)\\ (1,1,2,5) not possible\end{tabular}        & \begin{tabular}[c]{@{}l@{}}(1,2,3,4) is same as (4,3,2,1)\\ (1,1,2,4) is possible\end{tabular}     \\ \hline
\end{tabular}
\end{center}


We use the Fundamental Theorem of Counting to count the number of ways we can choose $k$ objects from $n$ distinct objects as follows\\

\begin{enumerate}
    \item \textbf{Order matters and without replacement}\\
    Let $T$ be the task of choosing $k$ objects from $n$ distinct objects, where order matters and without replacement.  The task $T$ is the same is filling $k$ ordered spaces, and can be broken into a sequence of sub-tasks
    
$$T:T_1\rightarrow T_2\rightarrow T_3 \rightarrow \cdots \rightarrow T_k$$
where $T_i$ fills the $i$th spot in the $k$ ordered spaces. 

Using the fundamental theorem of counting the total number of ways of doing the task $T$ is 
$$n\times (n-1)\times (n-2) \times (n-3) \times \cdots \times (n-k+1)$$

This number is called "$n$ permute $k$" and is denoted by

$$\Perm{n}{k}=\frac{n!}{(n-k)!}$$

\end{enumerate}

(2) \textbf{Unordered Without Replacement}\\
The first step is to calculate the ordered arrangements, which is \begin{align*}
\Perm{n}{k}=\frac{n!}{(n-k)!}
\end{align*}
Note that each ordered arrangement can be rearranged $k!$ times since we select $k$ objects in total. Therefore, we need to get rid repeats by dividing by $k!$, which is \begin{align*}
\frac{\Perm{n}{k}}{k!}=\frac{n!}{(n-k)!k!}
\end{align*}
So we got\begin{align*}
\Comb{n}{k}=\binom nk:=\frac{n!}{(n-k)!k!}
\end{align*}
\hfill\\
\hfill\\
(3) \textbf{With Replacement}\\
The number of ways if choose k objects where order does matter and with replacement in n different things is simply \begin{align*}
&n\times n \times n \times \cdots\times n \\
=&n^k
\end{align*}
\hfill\\
\hfill\\
(4) \textbf{ With replacement and order does not matter}\\
We want number of unordered arrangements of size $k$ from $n$ objects with replacement. This can be reformulate as the number of ways to choose $k$ "walls" from $(n+k-1)$ choices, i.e \begin{align*}
\Comb{n+k-1}{k}=\binom{n+k-1}{k} \text{ ways}
\end{align*} 
\begin{ex}
We set $n=10,k=3,$ where $S=\{0,1,2,3,4,5,6,7,8,9\}$.\\
Then we will have $n+k-1=12$ walls \begin{center}
 \begin{tabular}{|l|l|l|l|l|l|l|l|l|l|l|l|}
\hline
 &  &  &  &  &  &  &  &  &  &  &  \\ \hline
\end{tabular}
\end{center}
Then, for the event $(1,1,2)$, we will have (note that X means the wall that we choose) \begin{center}
\begin{tabular}{|l|l|l|l|l|l|l|l|l|l|l|l|}
\hline
 1 & X & X & 2 & X & 3 & 4 & 5 & 6 & 7 & 8 & 9  \\ \hline
\end{tabular}
\end{center}
For the event $(0,0,7)$, we will have \begin{center}
\begin{tabular}{|l|l|l|l|l|l|l|l|l|l|l|l|}
\hline
 X & X & 1 & 2 & 3 & 4 & 5 & 6 & 7 & X&8 & 9  \\ \hline
\end{tabular}
\end{center}
for the event $(5,9,7)$, we will have \begin{center}
\begin{tabular}{|l|l|l|l|l|l|l|l|l|l|l|l|}
\hline
 1  & 2  & 3 & 4 & 5&X & 6 & 7 &X& 8 & 9&X  \\ \hline
\end{tabular}
\end{center}
\end{ex}

The \textbf{number of possible arrangement of size k from n objects}
\textbf{1. Without replacement and order matters} for each of the four possibilities is listed below \begin{center}
\begin{tabular}{l|l|l|}
\cline{2-3}
                                & Without Replacement                                                                                     & With Replacement  \\ \hline
\multicolumn{1}{|l|}{Ordered}   & \begin{tabular}[c]{@{}l@{}}$\Perm{n}{k}=\frac{n!}{n-k}$\\ is also called "n permute k"\end{tabular}     & $n^k$             \\ \hline
\multicolumn{1}{|l|}{Unordered} & \begin{tabular}[c]{@{}l@{}}$\Comb{n}{k}=\frac{n!}{(n-k)!k!}$\\ is also called "n choose k"\end{tabular} & $\Comb{n+k-1}{k}$ \\ \hline
\end{tabular}
\end{center}




\begin{ex}
Consider choosing $3$ digits from the set $\{1,2,3,4,5\}$. 
\begin{center}
\begin{tabular}{|c|c|c|}
\hline \rowcolor{lightgray}
 & order matters & order does not matter\\
\hline
\cellcolor{lightgray} with &(1,2,2) is allowed & (1,2,2) is allowed\\
\cellcolor{lightgray} replacement &  but is different from (2,1,2) &  and is the same as (2,1,2)\\
\hline
\cellcolor{lightgray} without &(1,2,2) is not allowed, & (1,2,2) is not allowed,\\
\cellcolor{lightgray} replacement & (1,2,3) is different from (1,3,2) &  (1,2,3) is the same as (1,3,2)\\
\hline
\end{tabular}
\end{center}

In this case, $n=5$ and $k=3$, and we can compute as follows
\begin{align*}
^n\text{P}_k = 
{}^5\text{P}_3 = \frac{5!}{(5-3)!} = 5\cdot 4 \cdot 3 = 60\,, \qquad\quad
^n\text{C}_k = {}^5\text{C}_3
=\frac{5!}{(5-3)!\, 3!} = \frac{60}{3!} = 10\,,
\end{align*}
\begin{align*}
n^k  = 5^3 = 125\,,
\qquad\quad
^{n+k-1}\text{C}_{n-1}=
{}^{7}\text{C}_{4} = \frac{7!}{4!\,3!} = 35\,.
\end{align*}
\begin{center}
\begin{tabular}{|c|c|c|}
\hline \rowcolor{lightgray}
 & order matters & order does not matter\\
\hline
\cellcolor{lightgray} with replacement & 125 & 35\\
\hline
\cellcolor{lightgray} without replacement & 60 & 10\\
\hline
\end{tabular}
\end{center}

\end{ex}



\chapter[Random Variables]{Random Variables}
\section{Random Variables: Examples and Definitions}
In practical applications we are typically interested in numerical features associated to an outcome of an experiment. 

\begin{ex}
    Suppose we toss a two sided coin three times. The sample space of this experiment is given by: 
    $$\mathcal{S} := \begin{Bmatrix}
         HHH, & HHT, & HTT, &TTT  \\
            & HTH, & THT,& \\
            &THH, & TTH&
    \end{Bmatrix}
    $$
    We can consider numerical features of an outcome in the sample space, for example:
    \begin{enumerate}
        \item How many heads in the outcome?
        \item How many tails in the outcome?
        \item Are there more heads than tails?
        \item Is the first toss in the outcome a head?
        \item Is the last toss a tails?
        \item How many tails till the first head?
    \end{enumerate}
\end{ex}

\begin{ex}
    Suppose we randomly sample a student from the campus. The sample space for this experiment is given by:
    $$\mathcal{S} : = \{ \text{All students on campus}\}$$
    We can consider numerical features of an outcome (a student from the campus) in the sample space, for example:
    \begin{enumerate}
        \item What is the height of the chosen student?
        \item What is the average daily commute time for the student?
        \item The number of credits the student is enrolled in?
        \item The number of electronic gadgets the student owns?
        \item Average number of people the student interacts with on a given day?
        \item Is the student an athlete?
    \end{enumerate}
\end{ex}
These questions provide answers to very specific numeric/quantitative features of the student. 

\begin{defn}[Random Variable]
    A random variable $X$ is a real valued function on the sample space $\mathcal{S}$. That is, 
    \begin{eqnarray*}
        X :& \mathcal{S} &\longrightarrow \mathbb{R}\\
        & \omega &\mapsto X(\omega)
    \end{eqnarray*}
    The range of $X$, that is the set of all possible values that $X$ can take will be denoted as $\mathcal{X}$
\end{defn}
Intuitively, a random variable measures a specific quantitative feature of the sample space outcome. Here is a schematic to consider: 
\begin{center}
\begin{tikzpicture}
\tikzset{every node/.style={font=\footnotesize}}

% Function arrow and labels
\node at (-1.5, 0) (domain) {$X : S$};
\node at (2.5, 0) (codomain) {$\mathcal{X} \subseteq \mathbb{R}$};
\draw[->, thick, violet] (-0.2, 0) -- (1.2, 0);

% Left note
\node at (-1.5, -1.5) [align=center] {\textbf{Original Sample Space} \\ abstract object \\ the elements/outcomes \\ might not be comparable};

% Right note
\node at (2.5, -2.2) [align=center] {\textbf{New Sample Space} \\ is a subset of the \\ set of real numbers \\ \\ 1. can use the order relationship in real numbers, \\ 2. can also use properties like \\ addition, multiplication, \\ calculus etc.};

% Braces

% Arrows from notes to labels
\draw[->,green] (-1.2, -0.3) -- (-1.2, -0.7);
\draw[->,cyan] (2.2, -0.3) -- (2.2, -0.7);
\draw[->,cyan] (2.2, -1.8) -- (2.2, -2.2);

\end{tikzpicture}
\end{center}

\begin{defn}[Events Associated to a Random Variable]
    Given a random variable $$X: \mathcal{S} \longrightarrow \mathbb{R}$$
    we can use the order relationship of $\mathbb{R}$ to consider the following "interesting events" for $X$:
    \begin{eqnarray*}
        \{a \le X \le b\} :=& \text{$X$ takes values in the interval $[a, b]$}\\
        \{a \le X \} :=& \text{$X$ takes values in the interval $[a, \infty)$}\\
        \{ X \le b\} :=& \text{$X$ takes values in the interval $[-\infty, b]$}\\
    \end{eqnarray*}
    We can also allow for strict inequalities, countable unions, complements, and countable intersections. 
    \\
    We will assume that we are able to calculate the probabilities associated to these events, and these will be denoted by $P(a\le X \le b), P(a \le X), P(X \le b)$, etc
\end{defn}


\begin{ex}Suppose the sample space $\mathcal{S}$ is all students on campus.

For $ \omega \in \mathcal{S}$, we define:
\[
X(\omega) := \text{height of } \omega \text{ (in inches)}
\]

What are the values of \( X \)?

\[
\mathcal{X} := (0, \infty)
\]

Some "interesting events" corresponding to $X$ can be:
\begin{eqnarray*}
    \{ 60 \leq X \leq 65 \} &:=& \{ \omega \in S \mid X(\omega) \in [60, 65] \}\\
&=& \{\text{set of all students whose height is between 60 and 65 inches.}\}
\end{eqnarray*}

\begin{eqnarray*}
    \{X \ge 72\} &:=& \{ \omega \in S \mid X(\omega) \in [72, \infty) \}\\
    &=&\{\text{all students whose height is greater than or equal to 72 inches}\} 
\end{eqnarray*}

    
\end{ex}
%%%%%%%%%%%%%%%%%%%% Example: toss a coin three times %%%%%%%%%%%%%%%%
\begin{ex}
Suppose we toss a coin three times, and we want to know if the first toss was a head?
    \begin{equation*}
    X(\omega) =
    \begin{cases}
    0 & \text{if first toss in } \omega \text{ is not H} \\
    1 & \text{otherwise}
    \end{cases}
    \end{equation*}


\begin{center}
\begin{tikzpicture}[scale=1.2]
\tikzset{every node/.style={font=\footnotesize}}

% Nodes for outcomes
\node at (0, 4) (HHH) {HHH};
\node at (0, 3.5) (HHT) {HHT};
\node at (0, 3) (HTH) {HTH};
\node at (0, 2.5) (HTT) {HTT};
\node at (0, 2) (THH) {THH};
\node at (0, 1.5) (THT) {THT};
\node at (0, 1) (TTH) {TTH};
\node at (0, 0.5) (TTT) {TTT};

% Nodes for outputs
\node at (4, 3.25) (1) {1};
\node at (4, 1.75) (0) {0};



% Lines for 0
\draw[green] (HHH) -- (1);
\draw[green] (HHT) -- (1);
\draw[green] (HTH) -- (1);
\draw[green] (HTT) -- (1);

% Lines for 1
\draw[orange] (THH) -- (0);
\draw[orange] (THT) -- (0);
\draw[orange] (TTH) -- (0);
\draw[orange] (TTT) -- (0);
\end{tikzpicture}
\end{center}

Another way of looking at it:
\begin{equation*}
\begin{aligned}
    \{ X = 0 \} := X^{-1}(0) & = \{ \text{all sample space outcomes on which } X \text{ takes value 0} \} \\
    & = \{ \text{THH, THT, TTH, TTT} \}
\end{aligned}
\end{equation*}
Similarly,
\begin{equation*}
\begin{aligned}
    & \{ X = 1 \} := X^{-1}(1) = \{ \text{HHH, HTH, HHT, HTT} \}
\end{aligned}
\end{equation*}

\begin{equation*}
\begin{aligned}
    S := \{ & \{ \text{HHH, HTH, HHT, HTT} \} \mapsto \{1\}, \{ \text{THH, THT, TTH, TTT} \} \mapsto \{ 0 \} \}
\end{aligned}
\end{equation*}

\end{ex}


\begin{ex} 
Suppose we tossed a coin three times, and we were interested in counting the number of heads in three tosses. We can work through this as follows:

\begin{equation*}
X(\omega) := \# \text{ of heads in three tosses.}
\end{equation*}

\begin{center}
\begin{tikzpicture}[scale=1.2]
\tikzset{every node/.style={font=\footnotesize}}

% Nodes for outcomes
\node at (0, 4) (HHH) {HHH};
\node at (0, 3.5) (HHT) {HHT};
\node at (0, 3) (HTH) {HTH};
\node at (0, 2.5) (HTT) {HTT};
\node at (0, 2) (THH) {THH};
\node at (0, 1.5) (THT) {THT};
\node at (0, 1) (TTH) {TTH};
\node at (0, 0.5) (TTT) {TTT};

% Nodes for outputs
\node at (4, 4) (3) {3};
\node at (4, 3) (2) {2};
\node at (4, 2) (1) {1};
\node at (4, 1) (0) {0};




% Lines for 3
\draw[blue] (HHH) -- (3);

% Lines for 2
\draw[cyan] (HHT) -- (2);
\draw[cyan] (HTH) -- (2);
\draw[cyan] (THH) -- (2);

% Lines for 1
\draw[green] (THT) -- (1);
\draw[green] (HTT) -- (1);
\draw[green] (TTH) -- (1);

% Lines for 0
\draw[orange] (TTT) -- (0);
\end{tikzpicture}
\end{center}

\begin{equation*}
\begin{aligned}
    & \{ X = 0 \} = \{ \text{TTT} \} \\
    & \{ X = 1 \} = \{ \text{HTT, THT, TTH} \} \\
    & \{ X = 2 \} = \{ \text{HHT, HTH, THH} \} \\
    & \{ X = 3 \} = \{ \text{HHH} \}
\end{aligned}
\end{equation*}

\begin{equation*}
\mathcal{S} = \left\{
\begin{aligned}
    & \{ \text{HHH} \} \quad && \mapsto 3 \\[4pt]
    & \{ \text{HHT, HTH, THH} \} \quad && \mapsto 2 \\[4pt]
    & \{ \text{HTT, THT, TTH} \} \quad && \mapsto 1 \\[4pt]
    & \{ \text{TTT} \} \quad && \mapsto 0 
\end{aligned} 
\right\}
\end{equation*}

\begin{center}
\begin{tikzpicture}[scale=1.2]
\tikzset{every node/.style={font=\footnotesize}}

% Ellipses and elements inside
\draw (0,0) ellipse (0.5cm and 0.5cm);
\draw (1.5,0) ellipse (0.5cm and 1.5cm);
\draw (3,0) ellipse (0.5cm and 1.5cm);
\draw (4.5,0) ellipse (0.5cm and 0.5cm);

\node at (0, 0) {HHH};

\node at (1.5, -0.5) {HHT};
\node at (1.5, 0) {HTH};
\node at (1.5, 0.5) {THH};

\node at (3, -0.5) {HTT};
\node at (3, 0) {THT};
\node at (3, 0.5) {TTH};

\node at (4.5, 0) {TTT};

%\node at (-1, 0) {$\mathcal{S} = \left\{ \right.$}; % Big left curly brace
%\node at (5.5, 0) {$\left. \right\}$}; % Big right curly brace

\node at (0, -2.5) {\{3\}};
\node at (1.5, -2.5) {\{2\}};
\node at (3, -2.5) {\{1\}};
\node at (4.5, -2.5) {\{0\}};

% Arrows
\draw[->,orange] (0, -1.5) -- (0, -2);
\draw[->,orange] (1.5, -1.5) -- (1.5, -2);
\draw[->,orange] (3, -1.5) -- (3, -2);
\draw[->,orange] (4.5, -1.5) -- (4.5, -2);

\end{tikzpicture}
\end{center}
\end{ex}
%%%%%%%%%%%%%%%%%%%%%%%%% end of example %%%%%%%%%%%%%%




\begin{defn}[Cumulative Distribution Function]
    The cummulative distribution function associated to a random variable $$X: \mathcal{S} \longrightarrow \mathbb{R}$$
    is defined as 
    \begin{eqnarray*}
        F_X :& \mathbb{R} &\longrightarrow \mathbb{R}\\
        & x &\mapsto P(X \le x)
    \end{eqnarray*}
    that is
    \begin{eqnarray*}
        F_X(x) := P(\text{$X$ takes values less than or equal to $x$}) 
    \end{eqnarray*}
\end{defn}

%%%%% Example: Toss a coin once 
\begin{ex}
    Suppose we toss a coin with $P(H) = p \in (0,1)$ once. Let $X$ be the random variable that tracks the outcome of the coint toss, that is 
    $$X(\omega) = \begin{cases}
        1 \quad \text{if $\omega = H$}\\
        0 \quad \text{if $\omega = T$}
    \end{cases}$$
    We now calculate the cumulative distribution function for $X$. 

    \begin{enumerate}
        \item If $x < 0$:
\begin{eqnarray*}
\{X \leq x\} &=& \{\text{event that is impossible}\} \\
&=& \emptyset \\
\end{eqnarray*}
Therefore we must have:
\begin{eqnarray*}
    F_X(x) &=& P(X \leq x)\\
    &=& P(\emptyset) \\
    &=& 0
\end{eqnarray*}
for all $x <0$. 
\item If $x = 0$, 
\begin{eqnarray*}
    F_X(0) &=&  P(X \leq 0) \\
&=& \underbrace{P(X < 0)}_{0} + \underbrace{P(X = 0)}_{P(T)=(1-p)} 
\end{eqnarray*}
Therefore,
$$F_X(0) = 1-p$$
\item If $x \in (0,1)$,
\begin{eqnarray*}
    F_X(x) &=& P(X \leq x) \\
&=& \underbrace{P(X \leq 0)}_{0} + \underbrace{P(X=0)}_{1-p} + \underbrace{P(0<X<x<1)}_{0} \\
\end{eqnarray*}
Therefore, 
$$F_X(x) = (1-p) \quad \text{for all } x \in (0,1)$$
\item If $x=1$, 
\begin{eqnarray*}
    F_X(1) &=&  P(X \leq 1) \\
&=& \underbrace{P(X < 1)}_{1-p} + \underbrace{P(X = 1)}_{P(H)=p} 
\end{eqnarray*}
Therefore, 
$$F_X(1) = 1 $$
\item If $x>1$, 
\begin{eqnarray*}
    F_X(x) &=&  P(X \leq x) \\
&=& \underbrace{P(X \le 1)}_{1} + \underbrace{P(X > 1)}_{0} 
\end{eqnarray*}
Therefore, 
$$F_X(x) = 1 \quad \text{for all } x > 1 $$

    \end{enumerate}



The cdf $F_X$ can be written as follows:

\[
F_X(x) = \begin{cases}
0 & \text{if } x < 0 \\
(1-p) & \text{if } x \in [0,1) \\
1 & \text{if } x \geq 1
\end{cases}
\]

The plot of $F_X(x)$:
\vspace{1em}

\begin{tikzpicture}
    % Axis
    \draw[thick, ->] (-1.5,0) -- (3,0) node[anchor=north] {$x$};
    \draw[thick, ->] (0,-0.5) -- (0,3) node[anchor=east] {$F_X(x)$};
    \draw[very thin, gray] (-1.5,-0.5) grid (3,3);

    % CDF steps
    \draw[blue, thick] (-1.5, 0) -- (0, 0);
    \draw[orange, thick] (0, 0) -- (0, 1.5);
    \draw[blue, thick] (0, 1.5) -- (2, 1.5);
    \draw[orange, thick] (2, 1.5) -- (2, 3);
    \draw[blue, thick] (2, 3) -- (3, 3);

    % Points
    \filldraw[white, draw=blue, thick] (0, 0) circle (2pt);
    \filldraw[blue] (0, 1.5) circle (2pt);
    \filldraw[white, draw=blue, thick] (2, 1.5) circle (2pt);
    \filldraw[blue] (2, 3) circle (2pt);

    % Labels
    % \node at (-0.1, 0.15) [anchor=east] {$0$};
    \node at (0, 1.2) [anchor=west] {$1-p$};
    \node at (2, 2) [anchor=west] {$p$};
    \node at (0, 3) [anchor=west] {$1$};
    \node at (0, -0.3) [anchor=north] {$0$};
    \node at (2, -0.3) [anchor=north] {$1$};

    % Braces and annotations
    % \draw[decorate, decoration={brace, amplitude=10pt}] (-0.1,3) -- (-0.1, 1.5) node [black, midway, xshift=-0.6cm] {$p$};

\end{tikzpicture}
\end{ex}

\begin{ex}Suppose we toss a fair coin ($P(H) = \frac{1}{2}$) three times, and let 
$$X(\omega) := \text{number of heads in $\omega$}$$
Values that $X$ can take are 
$$ \X \coloneq \{0,1,2,3\}$$ 
\\

We now calculate the cdf of $X$.

\begin{enumerate}
    \item If $x < 0$:
\begin{eqnarray*}
F_X(x) &= P(X \leq x) \\
&= 0
\end{eqnarray*}
Therefore 
$$ F_X(x) = 0 \quad \text{for $x<0$}$$

\item If $x = 0$:
\begin{eqnarray*}
F_X(x) &=& P(X \leq 0) \\
&=& \underbrace{P(X < 0)}_{0} + \underbrace{P(X = 0)}_{\underbrace{P(\text{zero heads in three tosses})}_{P(TTT) = (\frac{1}{2})^3}}\end{eqnarray*} 
Therefore  $$F_X(0) = \frac{1}{8}$$
\item If  $x \in (0,1)$:
\begin{eqnarray*}
F_X(x) &= P(X \leq x) \\
&= \underbrace{P(X \leq 0)}_{\frac{1}{8}} + \underbrace{P(0 < X \leq x)}_{0} 
\end{eqnarray*}
Therefore 
$$F_X(x) &= \frac{1}{8} \quad \text{for } x \in (0,1)$$
\item  If $x = 1$:
\begin{eqnarray*}
F_X(1) &= P(X \leq 1) \\
&=& \underbrace{P(X \leq 0)}_{\frac{1}{8}} + \underbrace{P(0 < X < 1)}_{0} + \underbrace{P(X = 1)}_{\underbrace{P(\text{exactly two heads in three tosses})}_{P(\{HTT,THT,TTH\})=\frac{3}{8}}} \\
&=& \frac{1}{8} + P(HTT, THT, TTH) = \frac{1}{8} + 3\left(\frac{1}{2}\right)^3 = \frac{1}{8} + \frac{3}{8}
\end{eqnarray*}

Therefore  
$$F_X(1) &= \frac{4}{8} = \frac{1}{2}$$

\item  If $x \in (1,2)$:
\begin{eqnarray*}
F_X(x) &=& P(X \leq x) \\
&=& \underbrace{P(X \leq 1)}_{\frac{1}{2}} + \underbrace{P(1 < X \leq x)}_{0} \\
\end{eqnarray*}
Therefore  
$$F_X(x) &= \frac{1}{2} \quad \text{for $x \in (1,2)$}$$

\item  If $x = 2$:
\begin{eqnarray*}
F_X(2) &= P(X \leq 2) \\
&=& \underbrace{P(X \leq 1)}_{\frac{1}{2}} + \underbrace{P(1 < X < 2)}_{0} + \underbrace{P(X = 2)}_{P(\text{exactly two heads in three tosses})} \\
&=& \frac{1}{2} + P(HHT, HTH, THH) = \frac{1}{2} + 3\left(\frac{1}{2}\right)^3 = \frac{1}{2} + \frac{3}{8} \\
\end{eqnarray*}
Therefore $$F_X(2) &= \frac{7}{8}$$

\item  If $x \in (2,3)$:
\begin{eqnarray*}
F_X(x) &= P(X \leq x) \\
&= \underbrace{P(X \leq 2)}_{\frac{7}{8}} + \underbrace{P(2 < X < 3)}_{0} \\
\end{eqnarray*}
Therefore
$$F_X(x) &= \frac{7}{8} \quad \text{for $x \in (2,3)$} $$

\item If  $x \geq 3$:
\begin{eqnarray*}
F_X(x) &=& P(X \leq x) \\
&=& \underbrace{P(X \leq 2)}_{\frac{7}{8}} + \underbrace{P(2 < X \leq 3)}_{\text{P(X = 3)}} \\
&=& \frac{7}{8} + P(HHH) = \frac{7}{8} + \left(\frac{1}{2}\right)^3 = \frac{7}{8} + \frac{1}{8} = 1
\end{eqnarray*}

\end{enumerate}

We can consolidate all the information about $F_X$ as follows:
\[
F_X(x) = \begin{cases}
0 & \text{if } x < 0 \\
\frac{1}{8} & \text{if } x \in [0,1) \\
\frac{1}{2} & \text{if } x \in [1,2) \\
\frac{7}{8} & \text{if } x \in [2,3) \\
1 & \text{if } x \geq 3
\end{cases}
\]

The plot of $F_X$ is given below

\vspace{1em}
\begin{center}
\begin{tikzpicture}[scale=1.5]
    % Define variables for y-coordinates
    \def\yscale{4}
    \def\ya{0.125*\yscale}
    \def\yb{0.5*\yscale}
    \def\yc{0.875*\yscale}
    \def\yd{1*\yscale}

    % Axis
    \draw[thick, ->] (-0.5,0) -- (3.5,0) node[anchor=north] {$x$};
    \draw[thick, ->] (0,-0.1) -- (0,\yscale+0.2) node[anchor=east] {$F_X(x)$};
    \draw[very thin, gray] (-0.5,-0.1) grid (3.5,1.2*\yscale);

    % CDF steps
    \draw[blue, thick] (-0.5, 0) -- (0, 0);
    \draw[orange, thick] (0, 0) -- (0, \ya);
    \draw[blue, thick] (0, \ya) -- (1, \ya);
    \draw[orange, thick] (1, \ya) -- (1, \yb);
    \draw[blue, thick] (1, \yb) -- (2, \yb);
    \draw[orange, thick] (2, \yb) -- (2, \yc);
    \draw[blue, thick] (2, \yc) -- (3, \yc);
    \draw[orange, thick] (3, \yc) -- (3, \yd);
    \draw[blue, thick] (3, \yd) -- (3.5, \yd);

    % Points
    \filldraw[white, draw=blue, thick] (0, 0) circle (1pt);
    \filldraw[blue] (0, \ya) circle (1pt);
    \filldraw[white, draw=blue, thick] (1, \ya) circle (1pt);
    \filldraw[blue] (1, \yb) circle (1pt);
    \filldraw[white, draw=blue, thick] (2, \yb) circle (1pt);
    \filldraw[blue] (2, \yc) circle (1pt);
    \filldraw[white, draw=blue, thick] (3, \yc) circle (1pt);
    \filldraw[blue] (3, \yd) circle (1pt);

    % Labels
    \node at (-0.1, \ya) [anchor=east] {$\frac{1}{8}$};
    \node at (0.9, \yb) [anchor=east] {$\frac{1}{2}$};
    \node at (1.9, \yc) [anchor=east] {$\frac{7}{8}$};
    \node at (-0.1, \yd) [anchor=east] {$1$};
    \node at (0, -0.1) [anchor=north] {$0$};
    \node at (1, -0.1) [anchor=north] {$1$};
    \node at (2, -0.1) [anchor=north] {$2$};
    \node at (3, -0.1) [anchor=north] {$3$};
\end{tikzpicture}
\end{center}
\end{ex}


It is possible to classify all functions that can arise as cumulative distribution functions of random variable as follows:
\begin{thm}[Classification of CDFs]
A function $F(x)$ is a cumulative density function if and only if the following three conditions are satisifies: \begin{enumerate}
\item $\lim_{x\to -\infty}F(x)=0,$  $\lim_{x\to \infty}F(x)=1$
\item $F(x)$ is a non decreasing function
\item $F(x)$ is right continuous. i.e $\lim_{x\to x_o^+}F(x)=F(x_o),$ $\forall x_o\in \R$
\end{enumerate}
\end{thm}


\section{Discrete Random Variables}

\begin{defn}[Discrete Random Variable]
    We say a random variable $X: \mathcal{S} \longrightarrow \mathbb{R}$ is \textbf{discrete} if the cumulative distribution function of $X$, $F_X$ is a step function. 
\end{defn}

\begin{defn}[Probability Mass Function]
    If $X$ is a discrete random variable, the probability mass function associated to $X$ is defined as 
    \begin{eqnarray*}
        p_X :& \mathbb{R} &\longrightarrow \mathbb{R}\\
        & x &\mapsto P(X = x)
    \end{eqnarray*}
    that is, 
    $$p_X(x) := P(\text{$X$ takes the value $x$})$$
\end{defn}
It is easy to observe that the only points where $p_X$ can be non-zero are the values, $\mathcal{X}$, that $X$ can take. 

\begin{defn}[Parameters for Discrete Variables]
Suppose $X$ is a discrete random variable with probability mass function $p_X$ and taking values in the set $\mathcal{X}$. We can consider the following parameters associated to $X$
\begin{enumerate}
    \item \textbf{Expected Value:}
    $$E(X) := \mu_X = \sum_{x\in \mathcal{X}} x\; p_X(x)$$
    \item \textbf{Statisticians Unconscious Law:} Given a function $g: \mathbb{R} \longrightarrow \mathbb{R}$, then 
    $$E(g(X)) := \sum_{x\in \mathcal{X}} g(x) \; p_X(x)$$
    \item \textbf{Variance:}
    \begin{eqnarray*}
        V(X) := \sigma^2_X =& E ((X-\mu_X)^2)\\
        =& \sum_{x\in \mathcal{X}} (x-\mu_X)^2\; p_X(x)
    \end{eqnarray*}
    
\end{enumerate}
    
\end{defn}


\section{Continuous Random Variables}
\begin{defn}[Continuous Random Variable]
We say a random variable $X: \mathcal{S} \longrightarrow \mathbb{R}$ is \textbf{continuous} if the cumulative distribution function of $X$, $F_X$ is a continuous function.
\end{defn}

\begin{defn}[Probability Density Function]
    We say that the function $f_X: \mathbb{R}\longrightarrow \mathbb{R}$ is the probability density function of the continuous random variable $X$ (with cdf $F_X$) if $f_X$ satisfies the following:
    $$ F_X(x) = \int_{-\infty}^x f_X(t) dt$$
\end{defn}

\begin{defn}[Parameters for Continuous Variables]
    Suppose $X$ is a continuous random variable with probability density function $f_X$. We can consider the following parameters associated to $X$
\begin{enumerate}
    \item \textbf{Expected Value:}
    $$E(X) := \mu_X = \int_{-\infty}^\infty x\; f_X(x)\rm{dx}$$
    \item \textbf{Statisticians Unconscious Law:} Given a function $g: \mathbb{R} \longrightarrow \mathbb{R}$, then 
    $$E(g(X))  := \int_{-\infty}^\infty g(x)\; f_X(x)\rm{dx}$$
    \item \textbf{Variance:}
    \begin{eqnarray*}
        V(X) := \sigma^2_X =& E ((X-\mu_X)^2)\\
        =& \int_{-\infty}^\infty (x-\mu_X)^2\; f_X(x)\rm{dx}
    \end{eqnarray*}
\end{enumerate}
\end{defn}


\begin{defn}[Identically Distributed Random Variables]
    We say two random variables $X$ and $Y$ are identically distributed if the corresponding cdfs $F_X$ and $F_Y$ are pointwise equal (almost) everywhere. 
\end{defn}

\begin{ex}
    Suppose we toss a fair coin five times, and $X$ tracks the number of heads in five tosses, and $Y$ tracks the number of tails in five tosses. Then, it is possible to show that even though $X$ and $Y$ are not equal as functions on the sample space, the cdfs $F_X$ and $F_Y$ are exactly the same. As a result $X$ and $Y$ will be identically distributed. 
    \\

    Would we be able to conclude the same if we were to toss an unfair coin?
\end{ex}
\input{Chapter03}
%\chapter[Continuous Random Variable]{Continuous Random Variable}

%\input{Chapter05}
%\input{Chapter06}
%\chapter[Point Estimation]{Point Estimation}
\section[Introduction to Point Estimation]{Introduction to Point Estimation}
One goal of statistic is to draw insights/inference about certain aspects of the population using sample data. For example, we might want to estimate the following \begin{enumerate}
\item Average GPA of students on campus
\item Average time spent on recreational activities by students at UMD
\item Median age of everybody affiliated to UMD
\item The median yearly income of people in the USA
\end{enumerate}
In each these situations, we need to identify\begin{enumerate}
\item \colorbox{pink}{population of interest}
\item a \colorbox{yellow}{characteristic of the population} that we are interested in
\end{enumerate}
\begin{defn}
Suppose \begin{align*}
\{X_1,X_2,X_3,\cdots,X_n\}
\end{align*}
to be a random sample coming from a fixed population. Then, a \textbf{point estimator} for the population is any statistic $\widehat{\theta}$ for the random sample $\{X_1,X_2,\cdots,X_n\}$, i.e \begin{align*}
\widehat{\theta}:\X^n\to \R,\text{\qquad}\X:=\text{ values of }X_i
\end{align*}
We typically expect the values of $\widehat{\theta}$ to be a sensible value of a certain population characteristic, which is a \textbf{population parameter} denoted by $\theta$
\end{defn}
\note \begin{enumerate}
\item The population parameter $\theta$ is fixed once we fix the population
\item If we have population data, i.e a census, then we can calculate the exact value of $\theta$
\item Usually the population is intractable, which means we need to resort the sample data to get an estimate for $\theta$
\item Suppose $\theta$ is a parameter of interest. Given an estimator $\widehat{\theta}$ for $\theta$, $\widehat{\theta}$ is a statistic depending on the random sample $\{X_1,X_2,\cdots,X_n\}$. Therefore, the estimate $\widehat{\theta}$ will change everytime sample data changes
\end{enumerate}
\begin{defn}
If $\widehat{\theta}$ is a point estimator for the population parameter $\theta$, then the relation of average value of $\widehat{\theta}$ and true value $\theta$, or \textbf{Bias of $\widehat{\theta}$} is \begin{align*}
\text{Bias}(\widehat{\theta})=E(\widehat{\theta}-\theta)
\end{align*}
It is also the \textbf{expected error} when we use $\widehat{\theta}$ to estimate $\theta$ using a random sample of size $n$
\end{defn}
\begin{defn}
We say $\widehat{\theta}$ is an \textbf{unbiased estimator} for $\theta$ if \begin{align*}
\text{Bias}(\widehat{\theta})=0\text{\qquad for all possible choices of }\theta 
\end{align*}
i.e $E(\widehat{\theta}-\theta)=0$ for every possible $\theta$
\end{defn}
\exercise Suppose we have estimators $\widehat{\theta_1},\widehat{\theta_2},\cdots,\widehat{\theta_k}$ estimating $\theta$. Among all the $\widehat{\theta_i},$ is there a notion of one estimator being better than theothers?
\section[Principle of Unbiased Estimation]{Principle of Unbiased Estimation}
\begin{ex}
Let population be $\{0,1\}$ and the population distribution be $\text{Bernolli}(p)$, i.e $P(X=1)=p$. Let $\{X_1,X_2,\cdots,X_n\}$ be a random sample. Recall that \begin{align*}
T_o:&=\text{ sample total}\\
&=X_1+X_2+\cdots+X_n\\
T_o&\sim \text{Bin}(n,p)
\end{align*}
Let's define \begin{align*}
\hat{p}:=\frac{T_o}{n}
\end{align*}
This is the sample proportion of successes in a sample of size $n$. And $\widehat{p}$ is an estimator for $p$, the true probability of getting a success in a single Bernoulli trial.\\
\hfill\\
Now, \begin{align*}
E(\widehat{p})&=E\left(\frac{T_o}{n} \right)=\frac{E(T_o)}{n}=\frac{n\cdot p}{n}=p
\end{align*}
Therefore, $\widehat{p}$ is an unbiased estimator for $p$
\end{ex}
\begin{ex}
Suppose $\{X_1,X_2,\cdots,X_n\}$ is a random sample from a population with mean $\mu$ and variance $\sigma^2$. Then, we define the following: \begin{align*}
\widehat{\theta_1}&:=\overline{X}=\frac{\sum_{i=1}^nX_i}{n} &\text{sample mean}\\
\widehat{\theta_2}&:=\overline{X}+X_1-X_m& \text{sample mean plus 1st and second random sample}
\end{align*}
Then, \begin{align*}
E(\widehat{\theta_1})&:=E(\overline{X})=\mu\\
E(\widehat{\theta_2})&:=E(\overline{X}+X_1-X_m)\\
&=\mu+\mu-\mu = \mu
\end{align*}
Therefore, both $\widehat{\theta_1}$ and $\widehat{\theta_2}$ are biased estimations for he population mean $\mu$. Therefore, unbiased estimation does not guarantee a unique choice of estimator.
\end{ex}
\begin{defn}
Suppose $\widehat{\theta}$ is a statistic for a random sample $\{X_1,X_2,\cdots,X_n\}$ such that \begin{align*}
E(\widehat{\theta})=\theta \text{\qquad for all possible choices of }\theta 
\end{align*}
Then, $\sigma_{\theta}^2:=$ \textbf{variance of the sampling distribution of }$\widehat{\theta}$
\end{defn}
\begin{thm}
\textbf{Principle of Minimum Variance Unbiased Estimation:}\\
Among all unbaised estimators of $\theta$,  the one that has the minimum variance is called the \textbf{Minimum Variance Unbiased Estimator} of $\theta$
\end{thm}
\begin{ex}
We know that $\widehat{\theta_1}=\overline{X}$ and $\widehat{\theta_2}=\overline{X}+X_1-X_n$ are both unbiased estimators of $\mu$\\
To rank them, we calculate $V(\widehat{\theta_1})$ and $V(\widehat{\theta_2})$. \begin{align*}
V(\widehat{\theta_1})&=V\left(\frac{X_1+X_2+\cdots+X_n}{n} \right)\\
&=V\left(\frac{X_1}{n}+\frac{X_2}{n}+\cdots+\frac{X_n}{n}\right)\\
&=\frac{1}{n^2}\left(\sigma^2+\sigma^2+\cdots+\sigma^2 \right)\\
&=\frac{n\sigma^2}{n^2}=\frac{\sigma^2}{n}
\end{align*}
Similarly, we can show that \begin{align*}
V(\widehat{\theta_2})&=V(\overline{X}+X_1-X_n)\\
&=\frac{\sigma^2}{n}+2\sigma^2\\
&=\frac{2n+1}{n}\sigma^2
\end{align*}
Since $\frac{2n+1}{n}>\frac{1}{n}$ for all $n\in \N$, therefore, $V(\widehat{\theta_2})>V(\widehat{\theta_1})$. Therefore, if we were to choose only between $\widehat{\theta_1}$ and $\widehat{\theta_1}$, we would go with $\widehat{\theta_1}$. However, this does not prove that $\overline{X}$ is the Minimum Variance Unbiased Estimator
\end{ex}
\begin{thm}
If $\{X_1,X_2,\cdots,X_n\}$ is a random sample from a normally distributed population, i.e N$(\mu,\sigma^2)$. Then the estimator $\overline{X}$ is the  Minimum Variance Unbiased Estimator for $\mu$.\\
And the \textbf{standard error} of a point estimator $\widehat{\theta}$ is $\sigma_{\widehat{\theta}}=\sqrt{V(\widehat{\theta})}$, which provides a measure or precision of the point estimator $\widehat{\theta}$, i.e the standard deviation of the sampling distribution of $\widehat{\theta}$
\end{thm}
\note \begin{enumerate}
\item If $\widehat{\theta}$ has a continuous distribution, then \begin{align*}
P(\widehat{\theta}=\theta)&=P(\text{The estimator }\widehat{\theta}\text{ takes the value }\theta \text{ the true parameter value})\\
&=0
\end{align*}
Even so, the point-estimator provides an exact estimate for $\theta,$ we have zero confidence that the calculated point estimate will equal $\theta$
\end{enumerate}
\section[Methods of Point Estimation]{Methods of Point Estimation}
Suppose Population has distribution $X$ and probability mass function or probability density function of $X$ is $f(x)$.\\
Let $\{X_1,X_2,\cdots,X_n\}$ be a random sample of size $n$ from population.\\
\hfill\\
Then for $k=1,2,3,\cdots$\begin{enumerate}
\item The $k$th population moment or the $k$th distribution moment is the expected values of the random variable $X^k$, i.e $E(X^k)$
\item The $k$th sample moment for the random sample $\{X_1,X_2,\cdots,X_n\}$ is \begin{align*}
\frac{X_1^k+X_2^k+\cdots+X_n^k}{n} \text{\qquad i.e\quad}\frac{\sum_{i=1}^nX_i^k}{n}
\end{align*}
\end{enumerate}
\subsection[The Method of Moments]{The Method of Moments}
If there are $m$ parameters: $\theta_1,\theta_2,\theta_3,\cdots,\theta_m$ tand we want to estimate using $\{X_1,X_2,\cdots,X_n\}$. It is the same to get $m$-equations by equating the first $m$ population moments to the first $m$ sample moments and pray we can solve these equations to get estimators $\widehat{\theta_1},\widehat{\theta_2},\cdots \widehat{\theta_n}$ for $\theta_1,\theta_2,\cdots,\theta_m$ respectively.\\
\begin{ex}
\textbf{Sampling From Exponential Distribution}\\
Suppose $\{X_1,X_2,X_3,\cdots,X_n\}$ be a random sample from $\text{exp}(\lambda)$. Note that there is only one parameter to estimate and therefore only need to calculate first moments.\\
\hfill\\
Then, the first population moment is $$E(X)=\frac{1}{\lambda}$$ and the first sample moment is $$\frac{1}{n}\left(\sum_{i=1}^nX_i \right)=\overline{X}$$, the sample mean. Then, equating the population moments to sample moments, we got \begin{align*}
\frac{1}{\lambda}=\overline{X} \implies \lambda=\frac{1}{\overline{X}}
\end{align*}
Therefore, method of moment estimator for $\lambda$ is $$\widehat{\lambda}=\frac{1}{\overline{X}}$$
\end{ex}
\begin{ex}
\textbf{Sampling From Normal Distribution}\\
Suppose $\{X_1,X_2,X_3,\cdots,X_n\}$ be a random sample from $N(\mu,\sigma^2)$. We want to estimate $\mu$ and $\sigma^2$ (i.e $k=2$)\\
First, calculate the $k$th population moments:\begin{align*}
k=1&\to E(X)=\mu\\
k=2&\to E(X^2)
\end{align*}
Recall that the identity $\sigma^2=V(X)=E(X^2)-(E(X))^2$, therefore, we got \begin{align*}
E(X^2)=\sigma^2+\mu^2
\end{align*}
Then,equating the population moments to sample moments:\begin{align*}
E(X)&=\mu =\frac{1}{n}\sum_{i=1}^nX_i=\overline{X}\\
E(X^2)&=\sigma^2+\mu^2=\frac{1}{n}\sum_{i=1}^n X_i^2
\end{align*}
Therefore, we got the moment estimators as the following \begin{align*}
\widehat{\mu}&=\overline{X}\\
\widehat{\sigma^2}&=\frac{1}{n}\sum_{i=1}^n X_i^2-\left( \frac{1}{n}\sum_{i=1}^nX_i\right)^2
\end{align*}
\end{ex}
\begin{ex}
\textbf{Sampling From Gamma Distribution}\\
Suppose $\{X_1,X_2,X_3,\cdots,X_n\}$ be a random sample from $\text{Gamma}(\alpha,\beta)$, i.e $\alpha$ is the shape and $\beta$ is the scale.\\
Recall that $E(X)=\alpha \beta,$ $V(X)=\alpha \beta^2$. Since $E(X^2)-E(X)^2=\alpha \beta^2$, then $E(X^2)=\alpha \beta^2+(E(X))^2$. Therefore, we got the following population moments:\begin{align*}
E(X)&=\alpha \beta\\
E(X^2)&=\alpha \beta^2+(\alpha \beta)^2\\
&=\alpha \beta^2(1+\alpha)
\end{align*}
Evaluating sample moments to population moments, we get\begin{align*}
\overline{X}&=\alpha \beta\\
\frac{\sum_{i=1}^nX_i^2}{n}&=\alpha \beta^2(1+\alpha)
\end{align*}
Then let $A=\overline{X}$ and $B=\frac{1}{n}\sum_{i=1}^nX_i^2$, we need to solve the following for $\alpha,\beta$\begin{align*}
A&=\alpha \beta\\
B&=\alpha \beta^2(1+\alpha)
\end{align*}
set $\beta =\frac{A}{\alpha}$, then \begin{align*}
B&=\alpha \cdot \left(\frac{A}{\alpha} \right)^2(1+\alpha)\\
&=\frac{A^2(1+\alpha)}{\alpha}
\end{align*}
Therefore, \begin{align*}
&\alpha \beta =A^2+A^2\alpha\\
\implies &\alpha(B-A^2)=A^2\\
\implies &\alpha =\frac{A^2}{(B-A^2)}
\end{align*}
Therefore, the method of moment estimators are \begin{align*}
\widehat{\alpha}&=\frac{\overline{X}^2}{\frac{1}{n}\sum_{i=1}^nX_i^2-\overline{X}^2}\\
\widehat{\beta}&=\frac{\overline{X}}{\widehat{\alpha}}
\end{align*}
\end{ex}
\subsection[Maximum Likelihood Estimation]{Maximum Likelihood Estimation}
\begin{defn}
Suppose $\{X_1,X_2,\cdots,X_n\}$, a random sample from population with probability density function/probability mass function $f(x)$.\\
Let $f(x_1,x_2,\cdots,x_n)$ be the joint probability density function/probability mass function of $X_1,X_2,\cdots,X_n$. Suppose $X_i$'s are independent, then \begin{align*}
f(x_1,x_2,\cdots,x_n)&=\prod_{i=1}^nf(x_i)
\end{align*} 
Now, given sample data $\{x_1,x_2,\cdots,x_n\}$, we want to estimate parameters $\theta_1,\theta_2,\cdots,\theta_n$. Then, the \textbf{Likehood Function} for sample data $\{x_1,x_2,\cdots,x_n\}$ is defined as \begin{align*}
f(x_1,x_2,\cdots,x_n;\theta_1,\theta_2,\cdots,\theta_n)
\end{align*}
\hfill\\
And \textbf{Maximum Likelihood Estimation} is finding choices of parameters $\widehat{\theta_1},\cdots,\widehat{\theta_n}$ which maximize the likehood function for the observed sample data $\{x_1,x_2,x_3,\cdots,x_n\}$. That is, find $\widehat{\theta_1},\cdots,\widehat{\theta_n}$ such that \begin{align*}
f(x_1,x_2,\cdots,x_n;\widehat{\sigma_1},\widehat{\sigma_2},\cdots,\widehat{\sigma_n})\geq f(x_1,x_2,\cdots,x_n;\theta_1,\theta_2,\cdots,\theta_n)
\end{align*}
for all possible values of $\theta_1,\theta_2,\cdots,\theta_n$.\\
\hfill\\
Then, $\widehat{\theta_1},\cdots,\widehat{\theta_n}$ is the \textbf{maximum likelihood estimators} for $\theta_1,\cdots,\theta_n$ respectively, using $\{x_1,x_2,\cdots,x_n\}$. If we substitute $X_i$ for $x_i$, we will get maximum likelihood estimators $\widehat{\theta_1},\cdots,\widehat{\theta_n}$ for $\theta_1,\theta_2,\cdots,\theta_n$ respectively.
\end{defn}
\begin{thm}
\textbf{Properties of Maximum Likelihood Estimators}:\begin{enumerate}
\item \textbf{Invariance Principle}: If $\widehat{\sigma_1},\widehat{\sigma_2},\cdots, \widehat{\sigma_n}$ are maximum likelihood estimators for $\sigma_1,\cdots,\sigma_n$, then $h(\widehat{\sigma_1},\cdots,\widehat{\sigma_n})$ is an maximum likelihood estimators for $h(\sigma_1,\cdots,\sigma_n)$
\item \textbf{Large Sample Behavior}: When sample size is large, the maximum likelihood estimators $\widehat{\sigma}$ for $\sigma$ is atleast approximate unbiased, i.e $\widehat{\sigma}\approx \sigma$ and has variance that is either small as (or nearly as small as) can be achieved by any estimator.
\end{enumerate}
\end{thm}
\begin{ex}
\textbf{Sampling From Exponential Distribution}:\\
Suppose $\{X_1,X_2,X_3,\cdots,X_n\}$ be a random sample from $\text{exp}(\lambda)$. Then the probability density function of $X_i$ is $f(x_i;\lambda)=\lambda e^{-\lambda x_i}$.\\
\hfill\\
Then the likelihood function is \begin{align*}
f(x_1,x_2,\cdots,x_n;\lambda)&=\prod_{i=1}^n f(x_i;\lambda)\\
&=\lambda e^{-\lambda x_1}\cdot \lambda e^{-\lambda x_2}\cdots \lambda e^{-\lambda x_n}\\
&=\lambda^n e^{-\lambda(x_1+x_2+\cdots+x_n)}
\end{align*}
Therefore, \begin{align*}
f(x_1,x_2,\cdots,x_n;\lambda)&=\lambda^n \text{ exp}(-\lambda \sum_{i=1}^nx_i)
\end{align*}
Then, we want to maximize this as a function of $\lambda$'s. \\
If $g(\lambda)=\lambda^n \text{ exp}(-\lambda \sum_{i=1}^nx_i)$, then \begin{align*}
\ln(g(\lambda))&=\ln(\lambda^n\text{ exp}(-\lambda \sum_{i=1}^nx_i))\\
&=\ln(\lambda^n)+\ln(\text{exp}(-\lambda\sum_{i=1}^nx_i))\\
&=n\ln (\lambda)-\lambda\sum_{i=1}^n x_i
\end{align*}
This is log likelihood function, which is easier to maximize.\\
Let $h(\lambda)=\ln (g(\lambda))$, then \begin{align*}
h'(\lambda)&=\frac{d}{d\lambda}\left(n\ln (\lambda)-\lambda \sum_{i=1}^n x_i \right)\\
&=n\cdot  \frac{1}{\lambda}-\sum_{i=1}^n x_i
\end{align*} 
Therefore, \begin{align*}
h'(\lambda)=0\implies n\cdot \frac{1}{\lambda}=\sum_{i=1}^n x_i\implies \lambda =\frac{1}{\frac{1}{n}\sum_{i=1}^n x_i}
\end{align*}
Also, $h''(\lambda)=\frac{-n}{\lambda^2}<0\implies \lambda =\frac{1}{\frac{1}{n}\sum_{i=1}^n x_i}$ is a point of local maxima.\\
\hfill\\
Since $\ln$ is an increasing function, then $\lambda =\left(\frac{1}{n}\sum_{i=1}^nx_i \right)^{-1}$ maximizes $h(\lambda)\implies \lambda = \left(\frac{1}{n}\sum_{i=1}^nx_i \right)^{-1}$ maximizes the likelihood function $g(\lambda)$. Therefore, the maximum likelihood estimator for $\lambda$ is $\widehat{\lambda}=\frac{1}{\overline{X}}$, and this is the same as the moment estimator for $\lambda$
\end{ex}
\begin{ex}
\textbf{Sampling From Normal Distribution}\\
Suppose $\{X_1,X_2,X_3,\cdots,X_n\}$ be a random sample from $N(\mu,\sigma^2)$. Then $X_i$ has the following probability density function \begin{align*}
f(x_i;\mu,\sigma^2)&=\frac{1}{\sqrt{2\pi} \sigma}\cdot e^{-\frac{(x_i-\mu)^2}{2\sigma^2}}
\end{align*} 
Then, the likelihood function is given as \begin{align*}
f(x_1,x_2,\cdots,x_n;\mu, \sigma^2)&=\prod_{i=1}^n f(x_i;\mu, \sigma^2)\\
&=\left(\frac{1}{\sqrt{2\pi}\sigma} \right)^n\cdot e^{-\sum_{i=1}^n\left(\frac{(x_i-\mu)^2}{2\sigma^2} \right)}
\end{align*}
So that the log-likelihood function is \begin{align*}
\ln(f(x_1,x_2,\cdots,x_n;\mu,\sigma^2))&=\ln\left(\frac{1}{\sqrt{2\pi}\sigma} \right)^n+\ln \left( e^{-\sum_{i=1}^n\left(\frac{(x_i-\mu)^2}{2\sigma^2}\right)}\right)\\
&=\frac{n}{2}\ln \left(\frac{1}{2\pi \sigma^2} \right)-\frac{1}{2\sigma^2}\sum_{i=1}^n(x_i-\mu)^2
\end{align*}
Therefore, \begin{align*}
h(\mu, \sigma^2)&=\frac{n}{2}\ln\left(\frac{1}{2\pi \sigma^2} \right)-\frac{1}{2\sigma^2}\sum_{i=1}^n(x_i-\mu)^2
\end{align*}
Then, we need to maximize $h(\mu,\sigma^2)$ to calculate critical values, i.e calculate critical value where $\diffp{h}{\mu}=0 $ and $\diffp{h}{{{\sigma^2}}}=0 $. We can use Hessian matrix to show that these are max.\\
Then, we can show that the maximum likehood estimators are \begin{align*}
\widehat{\mu}&=\overline{X}\\
\widehat{\sigma^2}&=\frac{1}{n}\sum_{i=1}^n\left(X_i-\overline{X} \right)^2
\end{align*}
Note that while $\widehat{\mu}$ is unbiased, $\widehat{\sigma^2}$ is biased!
\end{ex}

\chapter[Statistical Inference]{Statistical Inference}

Statistical inference is the process of using data to make inferences about the underlying distribution that generated the data. All of the following can be phrased in these terms.

\ex{We are given a coin that comes up heads with some probability $p$. Estimate the value of $p$.
\label{example:estimate_p_coin_flip}
}

\ex{Estimate the CDF of heights for male students (at a fixed point in time, say this current semester) at UMD.}

\ex{A company claims that their new line of batteries last 100 hours under regular use. Is their claim accurate?}

\ex{Four fertilizers are available for use that, on paper, look as good as each other. Is this really the case?}
\hfill\\

In the following chapters, we try to develop techniques that answer some of these questions. We start with point estimators.

\section{Point estimators}

Point estimators are an attempt at estimating a quantity of interest about the population. We have already seen this idea in play: the sample average is how we estimate the population mean (i.e the sample average is a point estimator for the population mean) and the sample variance is a point estimator for the population variance.

\defn{
    Suppose that $\theta$ is a parameter. An \textit{estimator} for $\theta$, denoted by $\hat{\theta}$ (and $\hat{\theta}_n$ if we want to emphasize the sample size) is a statistic (i.e a function defined on the random sample).
}

\ex{
\[
    \overline{X} = \frac{X_1 + \ldots + X_n}{n}
\]
\[
T = \frac{X_1^2 + X_2^2 + \ldots + X_n^2}{n^2 + 1}
\]
$\overline{X}$ and $T$ are both statistics, and hence qualify as estimators for $E(X)$.
}\\

Once a \textit{random sample has been realized} (i.e once observations are made) we can use the observed values along with the estimator to \textit{create} an estimate. For example, suppose $n=3$, $x_1 = 1$, $x_2 = 2$, and $x_3 =3$ in the above. Then
\begin{gather*}
    \overline{x} = \frac{1 + 2 + 3}{3} \\
    t = \frac{1^2 + 2^2 + 3^3}{3} = \frac{1+4+9}{3} = \frac{14}{3}
\end{gather*}

The above example seems a little strange. We know $\overline{X}$ has been used repeatedly to estimate $E(X)$ but we have never used $T$. The reason is that $\overline{X}$ has mathematical properties that make it \textit{useful} in estimating $E(X)$ while $T$ does not have these properties. This shows that simply being an estimator is likely not enough to be a \textit{useful} estimator. We need something more.

\subsection{Unbiasedness, Consistency and Mean Squared Error}

We now turn our attention towards what makes a good (useful) estimator. There are several notions that are interrelated. We look at a couple below.

Let $X_1,...X_n$ be a sample of size $n$. Let $\hat{\theta}_n$ be an estimator for $\theta$ (i.e. $
\hat\theta_n$ is a statistic). Now $\hat\theta_n$ is said to be: 
\begin{enumerate}
    \item is said to be \textit{unbiased} if $E(\hat\theta_n) = \theta$ for all possible values of $\theta$.
    \item has a \textit{mean square error} (MSE) given by \[
\text{MSE}(\hat{\theta}_n) = E\left((\hat{\theta}_n - \theta)^2\right)
\]

    \item is said to be \textit{consistent} if for every $\varepsilon > 0$, 
    \[
    \lim_{n\rightarrow \infty} P(|\hat{\theta}_n - \theta|> \varepsilon) = 0
    \]

\end{enumerate}

\note
\begin{itemize}
    \item Unbiasedness indicates that we get the correct answer on average. 
    \item Consistency means that as the sample size (denoted by $n$) increases, the probability of making a \textit{significant error} (measured by $\varepsilon$) decreases towards zero.  
    \item The mean squared error measures how much the estimator deviates from the actual value on average. Given two (or more) estimators the one with the least mean squared error can be deemed to be ``good". 
    \item In order to keep the notation somewhat simple, in many situations we will abuse notation slightly and drop the subscript $n$ which is used to emphasize the sample size. For example, we may talk about $\overline{X}$ being a consistent estimator for $\mu$ when it is more precise to talk about $\overline{X}_n$ being consistent for $\mu$.  
    \item $\overline{X}$ is both an unbiased and consistent estimator for $E(X) = \mu$. The latter half of the statement relating to $\overline{X}$ being consistent is called the weak law of large numbers. 
    \item $S^2$ (the sample variance) is an unbiased estimator for $\sigma^2$. In fact the (up to now mysterious) division by $n-1$ of $\sum_{i=1}^n (X_i - \overline{X})^2$ is related to this. However the more "natural estimator" $
T = \frac{1}{n} \sum_{i=1}^n \left(X_i - \overline{X}\right)^2
$ for $\sigma^2$ is a biased estimator.
    \item The principle of unbiased estimation states that we prefer unbiased estimators over once that are not.
\end{itemize}

Let $X_1, \ldots, X_n$ be a random sample with $E(X_1) = \mu$ and $V(X_1) = \sigma^2$. Let's show that $\overline{X}$ is an unbiased estimator for $\mu$.

\begin{align*}
    E(\overline{X}) 
    &= E\left(\frac{1}{n} \sum_{i=1}^n X_i\right) \\
    &= \frac{1}{n} E\left( \sum_{i=1}^n X_i\right) \\
    &= \frac{1}{n} \sum_{i=1}^nE\left(  X_i\right) \\
    &= \frac{1}{n} n \mu = \mu
\end{align*}

However, this is not the only unbiased estimator. \\

\exercise Show that $X_1$ and $\frac{3X_1 + 4X_2}{7}$ are also all unbiased estimators of $\mu$.\\

To determine which of these estimators is better, we can turn towards comparing their mean squared errors. But before we look at the details, it helps to develop a little bit more terminology and results that relate the concepts of unbiasedness and mean squared error together. 

\defn{The quantity
$E(\hat{\theta}) - \theta$
is called the \textit{bias} of the estimator $\hat{\theta}$. We will denote it by $\text{bias}(\hat{\theta})$.
}\\

We can now rewrite the mean squared error as follows:

\[
\text{MSE}(\hat{\theta}) = 
(\text{bias}(\hat{\theta}))^2 + V(\hat{\theta})
\]

For unbiased estimators which have a bias of $0$, the equation becomes
\[
\text{MSE}(\hat{\theta}) = 
V(\hat\theta) 
\]

So computing the mean squared errors we obtain: 
\begin{align*}
    V(\overline{X}) 
    &= V\left(\frac{1}{n}\sum_{i=1}^n X_i\right) \\
    &= \frac{1}{n^2} V\left(\sum_{i=1}^n X_i\right) \\
    &= \frac{1}{n^2} \sum_{i=1}^n V(X_i) \text{ (as } X_i \text{ are independent)} \\
    &= \frac{1}{n^2} n\sigma^2 = \frac{\sigma^2}{n}\\
    \\
    V(X_1) &= \sigma^2 \\
    \\
    V\left(\frac{3X_1 + 4X_2}{7}\right) &=
    \frac{1}{49} \left(9V(X_1) + 16 V(X_2)\right) \\
    &= \frac{25}{49} \sigma^2 \approx \frac{\sigma^2}{2}
\end{align*}

So, for a larger sample (in fact, for $n\geq 3$ in this example), 
\[
V(\overline{X}) < V\left(\frac{3X_1 + 4X_2}{7}  \right) < V(X_1)
\]

Thus $V(\overline{X})$ seems like the better choice.

\defn{
The \textit{MVUE (minimum variance unbiased estimator)} tells us that When we have multiple unbiased estimators, we should pick the one with the lowest variance. 
}\\



\section{Constructing Estimators}

So far we have discussed what makes a point estimator good. What we have not done is discuss how to construct a point estimator. In this section, we talk about how to construct point estimators that under certain technical conditions have good properties.\\

\textit{The setup:} We have a random sample $X_1, \ldots, X_n$ with the common probability mass function/probability density function $f(x; \theta)$.

\subsection{The method of moments}

We start our discussion with the method of moments (abbreviated as \textbf{MM}). The method of moments relies on:
\begin{enumerate}
    \item $\frac{1}{n}\sum_{i=1}^n X_i^k$ being a good point estimator for the \textit{$k$th moment} $E(X^k)$, where $X$ is an R.V with the population distribution. This is due to the law of large numbers.
    \item Exploiting various relationships between the parameters and the moments.
\end{enumerate}

\ex{
Suppose $X_1, \ldots, X_n$ is a random sample with $X_i \sim Ber(p)$. $p$ is a fixed but unknown number. Construct a point estimator $\hat{p}_{MM}$ using the method of moments.\\

Let $X$ be a random variable with the same population distribution as the $X_is$, i.e $X \sim Ber(p)$. Now $E(X) = p $. Also, $\frac{1}{n}\sum_{i=1}^n X_i$ is a point estimator for $E(X)$, thus the first moment $E(X)$ can be estimated by $\frac{1}{n}\sum_{i=1}^n X_i$. But since $p = E(X)$, then $p$ can be estimated by
\[
\hat{p}_{MM} = \frac{1}{n} \sum_{i=1}^n X_i
\]
}

\ex{
Let $X_1, \ldots, X_n$ be a random sample with $X_i \sim \mathcal{N}(\mu,\sigma^2)$, where $\mu$ and $\sigma^2$ are fixed but unknown numbers. Construct point estimators $\hat{\mu}_{MM}$ and $\hat{\sigma}^2_{MM}$ for $\sigma^2$.\\

Let $X \sim \mathcal{N}(\mu, \sigma^2)$. Now $E(X) = \mu$ and $V(X) = \sigma^2$. But $V(X) = E(X^2) - [E(X)]^2$. Note that $\frac{1}{n}\sum X_i$ is a point estimator for $E(X)$, and $\frac{1}{n} \sum X_i^2$ is a point estimator for $E(X^2)$. Thus we can construct the method of moments estimators
\begin{gather*}
    \hat{\mu}_{MM} = \frac{1}{n} \sum_{i=1}^n X_i \\
    \hat{\sigma}^2_{MM} = \frac{1}{n} \sum_{i=1}^n X_i^2 - \left(\frac{1}{n} \sum_{i=1}^n X_i\right)^2
\end{gather*}
\label{example:norm_MM}
}

In algorithmic form, the method of moments can be written down as:

\begin{enumerate}
\item Step 1: Given parameters $\theta_1, \ldots, \theta_k$, write as many moments $E(X), \ldots, E(X^m)$ as needed as functions of the parameters $\theta_i$.

\iffalse	\begin{center}
	\begin{tabular}{ c c c }
	 $E(X)$ & = & $h_1(\theta_1, \ldots, \theta_k)$\\ 
	 $E(X^2)$ & =& $h_2(\theta_1, \ldots, \theta_k)$\\  
	 $\vdots$ & & \\
	 $E(X^k)$ & = & $h_k(\theta_1, \ldots, \theta_k)$
	\end{tabular}
	\end{center} \fi

\item Step 2: Solve for $\theta_i$ in terms of the moments. \iffalse
	\begin{center}
	\begin{tabular}{ c c c }
	 $\theta_1$ & = & $\ell_1(E(X_1), \ldots, E(X^k))$\\
	 $\vdots$ & & \\
	 $\theta_k$ & = & $\ell_k(E(X), \ldots, E(X^k))$
	\end{tabular}
	\end{center} \fi

\item Step 3: Form the estimators $\hat{{\theta_i}}_{MM}$. \iffalse
	\[
		\hat{\theta_i}_{MM} = \ell_i\left(
			\frac{1}{n} \sum_{j=1}^n X_j,
			\frac{1}{n} \sum_{j=1}^n X_j^2,
			\ldots,
			\frac{1}{n} \sum_{j=1}^n X_j^k
		\right)
	\] \fi
\end{enumerate}

Based on our examples so far, it is tempting to conclude that $k=m$. However, this is not the case: If $X_1, \ldots, X_n$ be a random sample with $X_i \sim U[-\theta, \theta]$, then $E(X_i)=0$ so we actually need to consider $E(X_i^2)$ in order to construct an estimator for $\theta$.

\note
\begin{enumerate}
	\item Our example with a random sample $X_1, \ldots, X_n$ with $X_1 \sim Ber(p)$ is related to our example in the previous section of attempting to determine the probability with which a coin comes up heads (Example \ref{example:estimate_p_coin_flip}). A natural approach to estimating such a probability would be tossing the coin multiple times and then computing the estimate as
		\[
			\frac{
				\text{\# of heads}
			}{
				\text{\# of tosses}
			}
		\]
		This is exactly what the $\hat{p}_{MM}$ suggests that you do!
	\item The method of moments can yield biased estimators (for example, $\hat{\sigma}^2_{MM}$ in Example \ref{example:norm_MM} is biased).
	\item Under certain technical conditions, the method of moments gives a consistent estimator. These topics are beyond the scope of this class.
\end{enumerate}

\subsection{Maximum likelihood estimation}

We now turn our attention to maximum likelihood estimation (abbreviated as \textbf{MLE}). The idea behind maximum likelihood estimators is to find an estimator $\hat{\theta}$ that maximizes the probability of the observed sample.

Maximum likelihood has two steps.

\begin{itemize}
	\item \textit{Step I}: Given a random sample $X_1, \ldots, X_n$ with a common probability mass/probability density function $f(x;\theta)$, compute the joint probability mass/probability density function. We typically write:
	\[
			\prod_{i=1}^n f(x; \theta)
	\]
	Here, we are using the independence of the $X_i$'s to write the joint pmf/joint pdf as a product of the invidual pmfs/pdfs.

	\item \textit{Step II}: Define the likelihood function:
	\[
		\mathcal{L}(\underline{\theta}) = \prod_{i=1}^n f(x_i;\underline{\theta})
	\]
	Find the $\hat{\underline{\theta}}$ which maximizes $\mathcal{L}(\underline{\theta})$.
\end{itemize}

Notice that we are thinking of the joint pmf/joint pdf as a function of $\underline{\theta}$. The idea is that when we are given observations $x_1, \ldots, x_n$ \textit{they are fixed} values. We seek to find the $\underline{\theta}$ that maximizes $\mathcal{L}(\underline{\theta})$, where $\underline{\theta}$ is the parameter (or collection of parameters) with values which are unknown to us. Further, step II requires us to optimize $\mathcal{L}(\underline{\theta})$. This suggests that calculus techniques will come into play.

\ex{
	Let $X_1, \ldots, X_n$ be a random sample with $X_1 \sim Ber(p)$. We find the MLE for $p$.

	Note that given $X\sim Ber(p)$, the probability mass is given by
	\[
		f(x;p) = p^x (1-p)^{1-x}
	\]
	for $x=0,1$ and 0 otherwise.

	Let's take $\theta = p$. Now we have
	\[
		\mathcal{L}(\theta) = \prod_{i=1}^n \theta^{x_i} (1-\theta)^{1-x_i}
	\]

	We want to find the value of $p$ that maximizes the above. A natural way to do this is by using calculus: Find the critical value for $\mathcal{L}(p)$. However, we will run into some computational difficulties if we try to compute $\mathcal{L}'(p)$ (we would need to use the product rule multiple times). We now pause the example for the moment and introduce the log-likelihood function that allows us to tackle this problem:
}
	\defn{
	The log-likelihood function $\ell(\underline{\theta})$ is defined as $\ell(\underline{\theta}) = \ln \mathcal{L} (\underline{\theta})$.
	}

	Note that: 
	\[
	\ell(\underline{\theta}) = 
	\ln \prod_{i=1}^n f(x_i, \underline{\theta}) =
	\sum_{i=1}^n \ln f(x_i;\theta)
	\]

	Thus, the \textit{log-likelihood is far easier to differentiate}, since the derivative of a sum is easier to compute than that of a product.\\

	\exercise Explain why finding a $\underline{\theta}$ that maximizes $\mathcal{L}(\underline{\theta})$ is equivalent to finding a $\underline{\theta}$ that maximizes $\ell(\underline{\theta})$. (Hint: $\ln(x)$ is an increasing function).\\

\textbf{Example continued. }

Now, we will write out the log-likelihood, take the derivative, and complete the derivation.
\begin{align*}
	\mathcal{L}(\theta) &= \prod_{i=1}^n \theta^{x_i} (1-\theta)^{1-x_i} \\
			    & \\
	\ell(\theta) 
	&= \ln \prod_{i=1}^n \theta^{x_i} (1-\theta)^{1-x_i} \\
	&= \sum_{i=1}^n \ln\left( \theta^{x_i} (1-\theta)^{1-x_i} \right)\\
	&= \sum_{i=1}^n \left( \ln\theta^{x_i}  + \ln (1-\theta)^{1-x_i} \right)\\
	&= \sum_{i=1}^n \left( x_i \ln\theta  +  (1-x_i)\ln (1-\theta) \right)\\
	& \\
	\ell'(\theta) 
	&= \frac{d}{d\theta} \left[\sum_{i=1}^n \left( x_i \ln\theta  +  (1-x_i)\ln (1-\theta) \right)\right]\\
	&= \sum_{i=1}^n \left(
		\frac{d}{d\theta}\left(
			x_i \ln\theta  +  (1-x_i)\ln (1-\theta)
		\right)
	\right)\\
	&= \sum_{i=1}^n \left(
			\frac{x_i}{\theta}  +  \frac{1-x_i}{1-\theta}(-1)
	\right)\\
	&= \sum_{i=1}^n \left(
			\frac{(1-\theta)x_i - \theta(1-x_i)}{\theta(1-\theta)}  
	\right)\\
	&= \sum_{i=1}^n \left(
			\frac{(1-\theta)x_i - \theta(1-x_i)}{\theta(1-\theta)}  
	\right)\\
	&= \sum_{i=1}^n \left(
			\frac{x_i - \theta x_i - \theta + \theta x_i}{\theta(1-\theta)}  
	\right)\\
	&= \sum_{i=1}^n \left(
			\frac{x_i - \theta}{\theta(1-\theta)}  
	\right)
\end{align*}

Setting $\mathcal{L}'(\theta)=0$ we obtain

\begin{gather*}
	\frac{1}{\hat{\theta}(1 - \hat{\theta})} \sum_{i=1}^n (x_i -\hat{\theta}) = 0 \\
	\left(\sum_{i=1}^n x_i\right) - n \hat{\theta} = 0 \\
	\hat{\theta} = \frac{1}{n} \sum_{i=1}^n x_i
\end{gather*}

Thus, the maximum likelihood estimate is $\hat{p}_{MLE} = \frac{1}{n} \sum_{i=1}^n x_i$ and the maximum likelihood estimator is
\[
	\hat{p}_{MLE} = \frac{1}{n} \sum_{i=1}^n X_i
\]

\ex{
	Suppose that $X_1, \ldots, X_n$ is an (iid) random sample with $X_1 \sim \mathcal{N}(\mu, \sigma^2)$ where both $\mu, \sigma^2$ are fixed but unknown. Find the MLEs for $\mu, \sigma^2$.\\

	Take $\theta_1 = \mu, \theta_2 = \sigma^2$. Then the pdf of a single random variable $X_i$ is given by:
	\[
		\frac{1}{\sqrt{2 \pi \theta_2}} e^{-\frac{(x - \theta_1)^2}{2\theta_2}}
	\]
	Now, we proceed to write the likelihood and log-likelihood.
	\begin{align*}
		\mathcal{L}(\theta_1, \theta_2) 
		&= \prod_{i=1}^n \frac{1}{\sqrt{2 \pi \theta_2}} e^{-\frac{(x_i-\theta_1)^2}{2\theta_2}}\\
		&\\
		\ell(\theta_1, \theta_2)
		&=
		\sum_{i=1}^n \ln\left(
			\frac{1}{\sqrt{2 \pi \theta_2}}e^{-\frac{(x_i-\theta_1)^2}{2\theta_2}}
		\right) \\
		&=
		\sum_{i=1}^n \left[
			-\frac{1}{2}\ln(2\pi\theta_2)
			+
			\ln e^{-\frac{(x_i-\theta_1)^2}{2\theta_2}}
		\right] \\
		&=
		\sum_{i=1}^n
			-\frac{1}{2}\ln(2\pi) 
			-\frac{1}{2}\ln(\theta_2)
			-\frac{(x_i-\theta_1)^2}{2\theta_2} \\
		&=
			-\frac{n}{2}\ln(2\pi) 
			-\frac{n}{2}\ln(\theta_2)
			-\frac{1}{2\theta_2}  \sum_{i=1}^n (x_i-\theta_1)^2
	\end{align*}
	Next, the critical points $(\hat{\theta_1}, \hat{\theta_2})$ of $\ell(\theta_1,\theta_2)$ should satisfy the equations
	\begin{gather*}
	\frac{\partial \ell}{\partial \theta_1} 
	\bigg \rvert_{(\theta_1, \theta_2) = (\hat{\theta_1},\hat{\theta_2})} = 0 \\
	\frac{\partial \ell}{\partial \theta_2} 
	\bigg \rvert_{(\theta_1, \theta_2) = (\hat{\theta_1},\hat{\theta_2})} = 0
	\end{gather*}
	Writing out the derivatives, we get
	\begin{align*}
		\frac{\partial \ell}{\partial \theta_1} 
		&= -\frac{1}{2\theta_2} \sum_{i=1}^n 2(x_i - \theta_1)(-1) \\
		&= \frac{1}{\theta_2} \sum_{i=1}^n (x_i - \theta_1) \\
		& \\
		\frac{\partial \ell}{\partial \theta_2}
		&= -\frac{n}{2\theta_2} + \frac{1}{2\theta_2^2} \sum_{i=1}^n (x_i - \theta_1)^2
	\end{align*}

	Solving for $(\hat{\theta}_1, \hat{\theta}_2)$ that simultaneously make $\frac{\partial\ell_1}{\partial \theta_1}$, $\frac{\partial \ell_2}{\partial \theta_2}$ equal 0 we obtain:
	\begin{gather*}
		\hat{\theta_1} = \frac{1}{n} \sum_{i=1}^n x_i \\
		\hat{\theta_2} = \frac{1}{n} \sum_{i=1}^n (x_i - \hat{\theta}_1)^2 = \frac{1}{n} \sum_{i=1}^n (x_i - \overline{x})^2
	\end{gather*}
}

So far, the method of moments and maximum likelihood estimators have both given us the same estimators. The following example illustrates this is not always the case.

\ex{
	Let $X_1, \ldots, X_n$ be a random sample with $X_1 \sim U[0,\theta]$ where $\theta > 0$ is fixed but unknown. Now
	\[\hat{\theta}_{MM} = \frac{2}{n} \sum_{i=1}^n X_i\]
	but on the other hand,
	\[
		\hat{\theta}_{MLE} = \max\{X_1, \ldots, X_n\}
	\]
}

\section{}
Under suitable conditions, maximum likelihood estimators are consistent as well. In addition, they are:
\begin{enumerate}
	\item Invariant: Given a function $g$, the MLE for $g(\theta)$ is $g(\hat{MLE})$.
	\item Asymptotically unbiased: Even if the MLE is biased, as the sample size increases, the bias goes to zero.
\end{enumerate}

There are further properties that we do not discuss here.

\section{A quick look back}

Now that we have discussed point estimators, here are some important facts about how we approached the topics.

\begin{enumerate}
	\item Values of parameters are \textit{fixed} but they are not known to us.
	\item When constructing estimators we took a \textit{parametric approach}, i.e we assumed that we knew the distribution of the population but not the value of the parameters. If these assumptions are incorrect, the information given by the estimators will not be useful.
 \item Notions of good for estimators depend on having good long run properties: on average, over a large sample (i.e. over a large number of observations from repeated repetitions). 
\end{enumerate}



\thischapterexercises

\exerciseitem Show that $S^2$ is an unbiased estimator for $\sigma^2$, but $T = \frac{1}{n} \sum (X_i - \overline{X})^2$ is not. \textit{Hint: $T= \frac{n}{n-1} S^2$, so $E(T)$ is easy to compute. The hard work goes into showing $S^2$ is unbiased.}\\

\exerciseitem Compute the bias, variance and mean squared error for $T= \frac{n}{n-1} S^2$. 

\exerciseitem Chebychev's inequality tells us that given any random variable $X$ with $E(X) = \mu$ and $V(X) = \sigma^2$, we have (for any $c > 0$)
\[
P\left(|X - \mu| \geq c\right) \leq \frac{\sigma^2}{c^2}
\]

Let $X_1, \ldots, X_n$,\ldots be i.i.d random variables that have the same distribution as $X$. Use this to prove the \textit{weak law of large numbers} for all such random variables. \\

\exerciseitem Consider Example \ref{example:estimate_p_coin_flip} where you are attempting to determine the probability $p$ of a coin coming up heads. Suggest an experiment to approximate the value of $p$. \\

\exerciseitem The following 10,000 numbers (hosted on ELMS) were generated from a Uniform distribution on $[0,\theta]$. Suggest an estimator $\hat{\theta}$ to estimate $\theta$ \textit{and} use $R$ to obtain the estimate based on the data. \\

\exerciseitem  The following 10,000 numbers (hosted on ELMS) were generated from an Exponential distribution, i.e random samples of the random variable $X \sim \text{Exp}(\lambda)$. Suggest an estimator $\hat{\lambda}$ to estimate $\lambda$ \textit{and} use $R$ to obtain the estimate based on the data.\\

\exerciseitem Consider a random variable $X$ with $E(X) = \mu$ and $V(X) = \sigma^2$. Let $X_1, \ldots, X_n$ be i.i.d, with each $X_i \sim X$ (i.e, a random sample of the variable $X$). Let
\[
T = k \sum_{i=1}^n i X_i
\]
You are given that $T$ is unbiased. What is the value of $k$?


\exerciseitem Suppose that $X_1, \ldots, X_n$ is a random sample with $X_1 \sim exp(\lambda)$. Compute the MM estimator and the MLE for $\lambda$. \\

\exerciseitem Suppose that $X_1, \ldots, X_n$ is a random sample where the pdf of $X_1$ is given by $f$ (defined below).
\[
	f(x;\theta) = 
	\begin{cases}
		\theta x^{\theta-1} & 0\leq x \leq 1 \\
		0 & \text{o.w}
	\end{cases}
\]
Let $\theta > 1$ be fixed. Compute the MM estimator and MLE for $\theta$.\\

\exerciseitem Let $X_1, \ldots, X_n$ be a random sample with $X_1 \sim U[0,\theta]$ where $\theta > 0$ is fixed but unknown.
\begin{enumerate}
	\item Compute $\hat{\theta}_{MM}$.
	\item Compute $\hat{\theta}_{MLE}$. 
	
	\textit{Hint: $\mathcal{L}(\theta) = \frac{1}{\theta^n}$. 
	To obtain the estimator, explain why $\theta \geq \max\{x_1, \ldots, x_n\}$. Also, explain why if we have $\theta_1 < \theta_2$, then we have $\mathcal{L}(\theta_1) > \mathcal{L}(\theta_2)$.} 

\item Which do you think is better in this case, $\hat{\theta}_{MM}$ or $\hat{\theta}_{MLE}$? Verify your guess empirically by writing a suitable program in $R$.
\end{enumerate}

%\chapter{Confidence Intervals}

In the previous chapter, we used a given sample $X_1, \ldots, X_n$ to construct estimates of various parameters of interest ($\overline{X}$ for $\mu$, $S^2$ for $\sigma^2$, etc.) In this chapter, we focus on using a sample to \textit{create intervals} that are likely to contain a parameter of interest.\\

\note It is the \textit{interval being constructed} that depends on the sample, \textbf{not the value of the parameter} (which while unknown to us is \textbf{fixed} and does not depend on the sample).

\defn{
A $1-\alpha$ confidence interval (abbreviated as \textbf{CI}) for a parameter $\theta$ is an interval 
\[
I(X_1, \ldots, X_n) = \left((a(X_1,\ldots, X_n), b(X_1, \ldots, X_n)\right)
\]
such that the equality 
\[
P(\theta \in I(X_1,\ldots, X_n)) = 1-\alpha
\]
holds for all possible values of $\theta$.
}

\note
\begin{itemize}
    \item We could require $P(\theta \in I(X_1, \ldots, X_n)) \geq 1-\alpha$.
    \item $\alpha$ is usually taken to be $0.01, 0.05$ or $0.1$.
\end{itemize}

\section{Constructing Confidence Intervals}

Let's start with an example to get an idea of what the process might look like. 

\ex{
Suppose that we have a random sample $X_1, \ldots, X_n$ with $X_1 \sim \mathcal{N}(\mu, \sigma^2)$. $\sigma^2$ is known, but $\mu$ is not known. We want to construct an interval $I(X_1, \ldots, X_n) = \left((a(X_1,\ldots, X_n), b(X_1, \ldots, X_n)\right)$ such that
\[
P(\mu \in I(X_1, \ldots, X_n)) 
=
P\left(
a(X_1, \ldots, X_n) 
< \mu <
b(X_1, \ldots, X_n)
\right)
= 0.99
\]
\label{example:CI_norm_knownSD}
}

It seems reasonable to try to use our knowledge of $\overline{X}$ in the construction. We start with the following fact:\\

\textbf{Fact. }If $Y_1, \ldots, Y_m$ are independent random variables such that for each $i$ we have $Y_i \sim \mathcal{N}(\mu_i, \sigma^2_i)$, then we have that
\[
\sum_{i=1}^m a_i Y_i \sim 
\mathcal{N}\left(
    \sum_{i=1}^m a_i \mu_i,
    \sum_{i=1}^m a_i^2 \sigma^2_i
\right)
\]\\

A consequence of this fact is that $\overline{X} \sim \mathcal{N}(\mu, \frac{\sigma^2}{n})$.\\

\note This is not an application of the CLT. This result is true even when the sample size is small $(n<30)$ and is exact (i.e, not an approximation) compared to our previous applications of the CLT.

Now note that from our previous knowledge of normal random variables,
\[
\frac{\overline{X} - \mu}{\sigma / \sqrt{n}} \sim \mathcal{N}(0,1)
\]

Thus, we have that
\begin{align*}
    P\left(-z_{0.005} < \frac{\overline{X}-\mu}{\sigma/\sqrt{n}}< z_{0.005} \right ) &= 0.99 \\
    P\left(-2.5758 < \frac{\overline{X} - \mu}{\sigma/\sqrt{n}} < 2.5758\right) &= 0.99
\end{align*}

\[
P\left(\overline{X} - 2.5758 \frac{\sigma}{\sqrt{n}} < \mu < \overline{X} + 2.5758\frac{\sigma}{\sqrt{n}}\right) = 0.99
\]

So, it seems reasonable to have the interval be 
\[
I(X_1, \ldots, X_n) = 
\left(
\overline{X} - 2.5758 \frac{\sigma}{\sqrt{n}},
\overline{X} + 2.5758 \frac{\sigma}{\sqrt{n}}
\right)
\]
as we have that $P(\mu \in I) = 0.99$.

Let's consider a more concrete example. 

\ex{
Take $n=10$ and $\sigma^2=4$, i.e we have a random sample $X_1, \ldots, X_{10}$ where $X_i \sim \mathcal{N}(\mu,4)$. Recall again that $\mu$ is fixed, but unknown to us. 

Suppose that we carry out an experiment and obtain a realization of the random variables $x_1, \ldots, x_n$. We calculate the sample average $\overline{X} = \frac{1}{10}\sum_i x_i$ and find that $\overline{X}=10$.

We can use the formula from the previous example (\ref{example:CI_norm_knownSD}) to calculate the CI (with confidence level $99\%$) for this realization:
\[
\left(
10 - 2.5758 \frac{4}{\sqrt{10}},
10 + 2.5758 \frac{4}{\sqrt{10}}
\right)
=
(6.7418, 13.2582)
\]
\label{example:CI_norm_knownSD_numerical}
}

\note

\begin{itemize}
    \item The results above heavily depended on the fact that $\overline{X} \sim \mathcal{N}(\mu, \frac{\sigma^2}{n})$. This in turn came from the fact that $X_i \sim \mathcal{N}(\mu, \sigma^2)$, where $\sigma^2$ is known.
    \item The value $0.99$ in the above equation gives us the \textit{confidence level}. In general we can specify the confidence level by $1- \alpha$ ($\alpha$ is called the \textit{significance level}). It is customary to choose $\alpha$ to be $0.01, 0.05$ or $1$ as was noted earlier.
    \item Regarding Example \ref{example:CI_norm_knownSD_numerical}, it is \textbf{not correct} to say that 
    \[
    P(\mu \in (6.7418, 13.2582)) = 0.99
    \]
    In fact, since $\mu$ is fixed, this probability is either 0 or 1 depending on whether $\mu$ falls inside the given interval. Confidence intervals should be interpreted using long-run frequencies.
\end{itemize}

\subsection{Approximate Confidence Intervals}

In the previous section we made two key assumptions.
\begin{itemize}
    \item The data is being generated from a normal distribution.
    \item The variance for this distribution is known.
\end{itemize}

These assumptions place strong constraints on constructing confidence intervals, even for $\mu$. It turns out that we can address these issues if we are willing to allow for approximate confidence intervals.

\defn{
We say that $I(X_1, \ldots, X_n)$ is an $1-\alpha$ approximate confidence interval for $\theta$ if 
\[
\lim_n P(\theta \in I(X_1, \ldots, X_n)) = 1-\alpha
\]
}

By setting
\[

\]
%\input{Chapter09}
\begin{appendices}
\section{Introduction}
		\noindent In the study of probability, we often run across situations where we need to talk about collections of outcomes as opposed to a single outcome. Such situations naturally lead us to the concept of \textit{events}, i.e. collections of outcomes that we would like to compute probabilities for. The language of sets allows us a convenient language for talking about such collections of events. 
		
		\begin{ex} 
			\begin{enumerate}
				\item In roulette, you have the option of betting on a collection of numbers that are ``close together" as opposed to making a betting on a single outcome. A person making bets in such a situation would need to be able to compute the probabilities for these more complicated bets as well.
				
				\item A researcher studying the effects of  a certain drug may be interested in the time it takes for the drug to break down in the blood stream. In such a situation, the researcher would end up with a range of time that it takes for the drug to break down (what is more realistic? that it takes five hours for the drug to breakdown in every single person who takes it or that it breaks down between 4.5 - 5.5 hours? ) 
			\end{enumerate} 
		\end{ex}
		
		Sets are essentially (well-defined) collections of objects. They are usually denoted by upper case letters. The objects that belong to a set are called its \textit{elements} or \textit{members}.  For example  $\{1,2,3,4,5,6\}$ is a set (with the curly braces indicating that we are looking at a set) and $1,2,3,4,5,6$ are all elements of $\{ 1,2,3,4,5,6 \}$. We use $\in$ to denote membership. So we can write $1\in\{1,2,3,4,5,6\}$ to indicate that 1 is a member of (or alternatively is an element of)  $\{1,2,3,4,5,6\}$.\\
		
		A sub-collection of objects from a set forms a \textit{subset} of that set: we say $A$ is a subset of $B$ if every element of $A$ is also an element of $B$. This fact is denoted by $A\subseteq{B}$. If $A$ is not a subset of $B$ (i.e. if there is some element of $A$ that is not in $B$) we write $A\nsubseteq B$ 
		\begin{enumerate}
			\item $\{1,2\}\subseteq\{1,2,5\}$ since $1,2$ are both elements of $\{1,2,5\}$
			\item $\{1,2\}\nsubseteq\{1,3,4\}$ since $2$ is not an element of $\{1,3,4\}$. Here $\nsubseteq$ should be read as ``not a subset of". We can denote $2$ is not an element of $\{1,3,4\}$ by writing $2\notin\{1,3,5\}$.
			\item For any set $C$, the empty set $\emptyset\subseteq C$ (where the empty set is the set that contains no elements .)
			\item For any set $C$, $C\subseteq C$
		\end{enumerate}
		
		\section{Set operations}
		As there are ways to combine numbers together to obtain a new number, there are ways to combine sets together to obtain different sets. We will discuss the following set operations: \begin{enumerate}
			\item Intersection 
			\item Union
			\item  Relative Complements
		\end{enumerate} 
		
		\subsection{Intersection}
		
		Given a collection of sets it is reasonable to ask what these sets have in common (this goes hand in hand with the word ``and"). The \textit{intersection} of a collection of sets is the set that contains the elements that the collection has in common.
		\begin{ex}
			Suppose that $A=\{1, 2, 3\}$, $B=\{2, 3, 4\}$, $C=\{3, 4, 5\}$ and $D=\{4, 5, 6\}$. Then the intersection of $A$ and $B$, denoted by $A\cap B$ is $\{2, 3\}$. Similarly $B\cap C=\{3, 4\}$ and $A\cap C=\{3\}$ etc. We can compute the intersection of three sets (or more) by using the same principles: $B\cap C\cap D=\{4\}$ as the only elements in common to all $B, C$ and $D$  is $4$.
		\end{ex} 
		
		It is possible to extend intersections to an infinite sequence of sets as well (it is possible to extend it even further but that is not going to be useful to us in the context of the class). Given a sequence $A_1,A_2,\ldots A_n,\ldots$ of sets, the intersection of the sets will be denoted by $\bigcap_{i=1}^\infty A_i$ and is the set that contains the elements common to all of the $A_i$.\\
		
		A collection of sets is said to be \textit{disjoint} if their intersection is empty. A collection of sets is said to be \textit{pairwise disjoint} if any two sets in the collection is disjoint. Any pairwise disjoint collection of sets is disjoint but the converse may not be true.  
		
		\begin{ex}
			Let $A_i=\{i\}$ for each positive integers $i$. So $A_1=\{1\},  A_2=\{2\}, A_3=\{3\}, \ldots$. The $A_i$ are pairwise disjoint as $A_i\cap A_j = \emptyset$ whenever $i\neq j$.\\
			
			The sets $A= \{1,2\}$, $B= \{1,3\}$ and $C= \{2,3\}$ are disjoint. They are not pairwise disjoint (in this example, the intersection of any two sets from $A, B, C$ is non-empty but $A\cap B \cap C=\emptyset$). 
		\end{ex}
		
		
		\noindent\textbf{Exercise:}  Suppose that $A_n=\{n, n+1, n+2, \ldots \}$ for all positive integers $n$. Explain why their intersection is empty, i.e. explain why $\bigcap_{n=1}^\infty A_n=\emptyset$
		
		\subsection{Union}
		
		Given a collection of sets it is also reasonable to ask what you can get by combining all the elements in these sets together (this goes hand in hand with the word ``or"). The \textit{union} of a collection of sets is the set that contains the elements that appear in \textit{at least one} of the sets in the collection.
		
		\begin{ex}
			Suppose that $A=\{1, 2, 3\}$, $B=\{2, 3, 4\}$, $C=\{3, 4, 5\}$ and $D=\{4, 5, 6\}$. Then the union of $A$ and $B$, denoted by $A\cup B$ is $\{1, 2, 3, 4\}$. Similarly $B\cup C=\{2, 3, 4, 5\}$ and $A\cup C=\{1, 2, 3, 4, 5\}$ etc. We can compute the unions of three sets (or more) by using the same principles: $B\cup C\cup D=\{2, 3, 4, 5, 6\}$. 
		\end{ex} 
		
		Similar to intersections, it is possible to extend unions to an infinite sequence of sets as well. Given a sequence $A_1,A_2,\ldots A_n,\ldots$ of sets, the union of the sets will be denoted by $\bigcup_{i=1}^\infty A_i$ and is the set that contains the elements appear in at least one $A_i$.\\
		
		
		\noindent\textbf{Exercise:}  Suppose that $A_n=[\frac{1}{n+1}, 1]$ for all positive integers $n$. Why is their union the interval $(0, 1])$? 
		
		\subsection{Relative Complements}
		
		Given sets $A, B$ another reasonable question to ask is what elements are unique to each set. Relative complements answer this question.\\ 
		
		The set of elements in $A$ but not in $B$ is denoted by $A-B$ and is read $A$ take away $B$.
		
	    \begin{ex}
	    	Suppose that $A=\{1,2,3\}$ and $B=\{2,3,4\}$. Then $A-B=\{1\}$ and $B-A=\{4\}$.
	    \end{ex} 
		
		Unlike union and intersection, there is no natural extension of the notion of a relative complement to a sequence of sets.
		
		The notion of complement of a set $A$ often denoted by $A^c$ or $A'$ will appear throughout our study of probability. $A^c$ stands for the set of elements not in $A$. This however causes issues: say $A=\{1,2\}$, then objects not in $A$ contains, among other things, the math building, your laptop etc. This is unlikely to be useful (in fact it is not even a set if we were to do a more technical study of sets ). To address this issue we introduce the concept of a  \textit{universal set}. The universal set is meant to contain ``everything of interest".    Complements are actually relative complements taken with the universal set. For example if the universal set $U=\{1,2,3,4,5,6\}$ and $A=\{1,2\}$ then $A^c=U-A = \{3,4,5,6\}$. \\ 
		
		\textit{ In the study of probability we take the universal set to be the sample space} as every outcome of interest is in the sample space.
		
		
		 
		 
		 \section{Venn diagrams and visualizing sets}
		 
		 Venn diagrams allow us a nice way to visualize sets. The idea is to use a rectangle to denote the universal set and to use a different shape (most of the time circles) to denote various subsets of interest. We think of the interior of the circles we have drawn as containing the elements from that set.\\
		 
		 Here are a couple of examples. 
		 
		 \begin{enumerate}
		 	\item This Venn diagram has a subset $A$ highlighted in gray.\\
		 	\begin{tikzpicture}[fill=gray]
		 		\draw (0,0) circle (1) (0,0)  node [text=black,above] {$A$}
		 		(-2,-2) rectangle (3,2) node [text=black,above] {$U$};
		 		\fill (0,0) circle (1) (0,0) node [text=black,above] {$A$};
		 		
		 	\end{tikzpicture} 
		 	\item This Venn diagram highlights $(A\cap B)^c$, i.e. the shaded area is the complement of the intersection of $A$ and $B$.\\
		 	\begin{tikzpicture}
		 		\filldraw[fill=gray] (-2,-2) rectangle (3,2) node [text=black,above] {$U$};
		 		\scope % A \cap B
		 		\clip (0,0) circle (1);
		 		\fill[white] (1,0) circle (1);
		 		\endscope
		 		% outline
		 		\draw (0,0) circle (1) node [text=black,above] {$A$}
		 		(1,0) circle (1) node [text=black,above] {$B$};
		 	\end{tikzpicture}
		 \end{enumerate}
		 
		 
		 
		\subsection{Some useful identities}
		 There are certain identities that are useful when working with sets. For us, these identities establish connections between the three set operations that we have mentioned in this note: union, intersection and taking (relative) complements. \\

		 
		 Here is a list of useful identities: \begin{enumerate}
		 	\item  $(A^c)^c=A$
		 	\item $(A\cap B)^c = A^c \cup B^c$
		 	\item $(A\cup B)^c = A^c \cap B^c$
		 	\item $(\bigcap_{i=1}^n A_i)^c = \bigcup_{i=1}^n A_i^c$
		 	\item $(\bigcup_{i=1}^n A_i )^c= \bigcap_{i=1}^n A_i^c$
		 	\item $(\bigcap_{i=1}^\infty A_i)^c = \bigcup_{i=1}^\infty A_i^c$
		 	\item $(\bigcup_{i=1}^\infty A_i)^c = \bigcap_{i=1}^\infty A_i^c$
		 \end{enumerate} 
		
		Venn diagrams can be useful in establishing some of these identities: Draw a Venn diagram for the set on the left hand side of the equation, do the same for the set on the right hand side of the equation and if the shaded regions are the same then the equation holds! 
		 
		
		 \section{The role of sets in Stat 400}
		 
		 As we discussed above, the idea for us that sets provide a \textit{useful language for discussing events and combinations of events}. The idea is that  for a given experiment that we are interested in, the set of all possible outcomes will act as the universal set, i.e. the role of the universal set will always be played by the sample space. Events can now be described as \textit{subsets} of the sample space. \\
		 
		 
		 Consider rolling a die. The possible outcomes are now $1, 2,3,4,5,6$. The set that contains these outcomes is now $\Omega=\{ 1,2,3,4,5,6 \}$ . Examples of events include \begin{enumerate}
		 		\item $\{1,2\}$: you see either a 1 or a 2 when the die is rolled.
		 		\item $\{4,5,6\}$: you see either a 4, a 5, or a 6 when the die is rolled.
		 		\item $\emptyset$: the empty or null event indicates the (im)possibility that nothing happens when you conduct the experiment.
		 		\item $\{1,2,3,4,5,6\}$ you see either a 1 or a 2 or 3 or a 4 or a 5 or a 6 when the die is rolled
		 		\end{enumerate}
		 
		 \textbf{In general given an event $A$ lists a collection of possible outcomes of the experiment of interest  and $P(A)$ denotes the probability that an outcome listed in $A$ takes place.} 

    \section{Cardinality}

    The last concept we will need from set theory is that of cardinality. Cardinality aims to measure how many elements are in a set. In order to define what it means for two sets to have the same cardinality, we need the notion of a bijection first.

    Recall that a function $f: A \rightarrow B$ is a \textit{bijection} if:
    \begin{enumerate}
        \item it is \textit{surjective} (or \textit{onto}), i.e for any $b\in B$ there is some $a \in A$ such that $f(a) = b$.
        \item it is \textit{injective} (or \textit{one-to-one}), i.e for any $a_1, a_2$ in $A$ with $f(a_1) = f(a_2)$, we have $a_1 = a_2$.
    \end{enumerate}
    
    \begin{defn}
        Two sets $A$ and $B$ have the same cardinality (denoted by $|A| = |B|$) if and only if there is a \textit{bijection} between $A$ and $B$.
    \end{defn}

   

    So $|A| = |B|$ implies that each element in $A$ can be paired with a unique element of $B$. \textit{Thus the number of elements must be the same as that of $B$.}
    \begin{ex}
        \begin{enumerate}
            \item Let $2 \mathbb{N} = \left\{2,4,6,8,10,\ldots\right\}$. Then
            \[
            |2\mathbb{N}| = |\mathbb{N}|
            \]
            The function $f: \mathbb{N} \rightarrow \mathbb{N}$
            \[
            f(x) = 2x
            \]
            is a bijection.
            \item The interval $(-\frac{\pi}{2}, \frac{\pi}{2})$ has the same cardinality as $\mathbb{R}$. The function $f: (-\frac{\pi}{2}, \frac{\pi}{2}) \rightarrow \mathbb{R}$ where
            \[
            f(x) = \tan(x)
            \]
            is a bijection.
        \end{enumerate}
    \end{ex}

    Note that in the example above, $2\mathbb{N} \subseteq \mathbb{N}$ and $\mathbb{N} \neq 2 \mathbb{N}$ (sometimes denoted as $2\mathbb{N} \varsubsetneq \mathbb{N}$), and $(-\frac{\pi}{2}, \frac{\pi}{2}) \varsubsetneq \mathbb{R}$. It turns out that this happens when we have infinite sets. This is not the case with finite sets.

    In our context, the notions of \textit{countable} and \textit{uncountable} sets are important: problems involving countable sets are usually handled using sums (and matrices), while those involving uncountable sets use integrals.

    \begin{defn}
        A set $A$ is said to be \textit{countable} if it is empty or there is a \textit{surjection}, or surjective function $f: \mathbb{N} \rightarrow A$. A set is \textit{uncountable} if it is not countable.
    \end{defn}
    Note that \textit{finite sets} are countable too. For example, if 
    \[
    A = \{0,1,2,3,4,5\}
    \]
    \[
    f: \mathbb{N} \rightarrow A
    \hspace{1cm}
    f(x) = x\mod6
    \]
    (where $x\mod6$ is the remainder when you divide $x$ by 6).
    
    We differentiate between countable sets that are finite and countable sets that are infinite by calling sets that are countable and infinite \textit{countably infinite}.

    Recognizing a set as finite or infinite is usually fairly straightforward. However recognizing an infinite set as countably infinite or uncountable can be difficult. Here, I will list out the most common countably infinite and uncountable sets.

    \textbf{Countable sets}
    \begin{itemize}
        \item $\mathbb{N}$
        \item $\mathbb{Q}$
        \item If $A_1, A_2, \ldots, A_n$ are countable, then $A=A_1 \times \ldots \times A_n$ (also written $\prod_{i=1}^n A_i$, the set of all ordered $n$-tuples) is also countable.
    \end{itemize}

    \textbf{Uncountable sets}
    \begin{itemize}
        \item $\mathbb{R}$
        \item Any nontrivial interval, e.g $(a,b)$ where $a < b$
        \item The irrationals
        \item If $A_1, \ldots, A_n$ are uncountable, then so is $A = A_1 \times \ldots \times A_n$.
    \end{itemize}

    If you are up for a challenge, see if you can use the following theorem to show that $\mathbb{N}$ and $\mathbb{N} \times \mathbb{N}$ have the same cardinality.

    \begin{thm} \textbf{Schr\"{o}der-Bernstein theorem.}
    
        Suppose that there are one-to-one functions $f: A \rightarrow B$ and $g: B\rightarrow A$. Then $|A| = |B|$.
    \end{thm}

\thischapterexercises
    \exerciseitem Use Venn diagrams to determine whether the following relations always hold.
          \begin{enumerate}
              \item $(A \cap B)^c = A^c \cup B^c$
              \item $(A - B) \cap C = (A \cup B) \cap (A \cup C)$
              \item $(A - B) \cap (B- A) = \emptyset$
          \end{enumerate}
    \\
    \exerciseitem Let $A = \{3n : n \in \N\}$. Find a bijection from $A$ to $\N$.\\
    
    \exerciseitem Let $\Omega = \N$, $A = 2\N$, $B = 3\N$. Describe the following sets.
    \begin{enumerate}
        \item $A \cap B$
        \item $A \cup B$
        \item $A - B$
        \item $B - A$
    \end{enumerate} \\
    
    \exerciseitem Let $\Omega = \{1,2,3,4,5,6\}$, $A = \{2,3,5\}$, $B = \{2,4,6\}$. Compute:
    \begin{enumerate}
        \item $A \cup B$
        \item $A \cap B$
        \item $A - B$
        \item $B - A$
    \end{enumerate}
\end{appendices}



\end{document}